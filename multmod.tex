\chapter{Multiple Module Support}
The next step is to implement multiple modules into the Haskell semantics. Similar to including files or objects in other programming languages, Haskell modules can include other modules and use functions, types, and typeclasses declared in the module.
There are a several considerations and additional checks that need to be made.
Modules need to include other modules
There cannot be inclusion cycles
Modules need to be able to access user defined types from the other modules that are included
When referencing types from other modules, there is a scope where
if the user makes another type of the same name in the current module
then when the user references the type, it refers to the type from the current module.
For instance:
\begin{lstlisting}
File 1:
module File1 where
{data A = B
}
File 2:
module File2 where
{data A = B
;data C = B A
}
\end{lstlisting}
	This will compile. The A used in data C is File2.A
However, If there are multiple types with the same name declared outside of the current module. If you try to refer to the type without the parent module, there will be a compiler error because there is ambiguity.
For instance:
\begin{lstlisting}
File 1:
module File1 where
{data A = B
}
File 2:
module File2 where
{data A = B
}
File 3:
module File3 where
{import File1
;import File2
;data C = B A
}
\end{lstlisting}

	This will not compile because in File 3, type A is ambiguous and can mean File1.A or File2.A
Type synonyms need to include polymorphism
For instance:
	The user can write
	type A a = B a

Since we need to check for module inclusion cycles and also build the set of user defined types for each module and included modules, I decided to use a tree.
The plan for the algorithm is as follows
1. Construct tree for module inclusion
2. Check tree for cycles
3. Go to each leaf and recursively go up the tree and build alpha* and beta* for the types of the module and the children and desugar the scope so that each type specifies the scope.

Where alpha is the map of type synonyms declared in the current module and alpha* is the map of type synonyms declared in the current module and all the included modules. Beta is the set of user defined types from using data and newtype declared in the current module. Beta* is the set of user defined types from using data and newtype declared in the current module and all the included modules. 
Desugar the scope means that when the user references a type, desugar the reference to also include the parent module at all times.
The syntax also needed to be changed to allow for multiple modules. The new syntax added is
\begin{lstlisting}
//  CUSTOM SYNTAX NOT PART OF OFFICAL HASKELL

    syntax ModuleList ::= Module [klabel('modListSingle)] | Module "<NEXTMODULE>" ModuleList [klabel('modList)]
\end{lstlisting}

This is because K cannot read mutiple files. So instead all the included modules for a program are dumped into one file and are seperated by the keyword <NEXTMODULE>
This creates a list of modules called ModuleList.