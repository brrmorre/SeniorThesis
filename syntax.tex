\chapter{Context Free Syntax}
The Haskell 2010 report is the current official specification of the Haskell language. %https://www.haskell.org/onlinereport/haskell2010/
The grammar specified in section 10.5 of the Haskell 2010 report is a specification of the expanded syntax of haskell.
%https://www.haskell.org/onlinereport/haskell2010/haskellch2.html#x7-210002.7
As specified in section 2.7, the expanded syntax of haskell specifies haskell programs when written using semicolons and braces. However, these can be omitted in a real haskell program. The compiler will then utilize layout rules for certain grammar structures instead. These are specified in section 10.3
%https://www.haskell.org/onlinereport/haskell2010/haskellch10.html#x17-17800010.3
The parser that I have written in K does not implement these layout rules and instead only can parse the expanded, layout insensitive syntax of haskell. It would require another script to convert a program written using the layout sensitive syntax into the expanded syntax in order to parse the program.
Haskell has a context free grammar.
%https://www.haskell.org/onlinereport/haskell2010/haskellch10.html#x17-18000010.5
	Section 10.1 specifies the notation used in the grammar.
%https://www.haskell.org/onlinereport/haskell2010/haskellch10.html#x17-17600010.1
The notation of 10.1 are always in bold in the grammar. So
\begin{lstlisting}
qvarid -> [ modid . ] varid
\end{lstlisting}
Means that ‘modid .’ is optional, and the brackets ‘[ ]’ are not part of the haskell code, but the period ‘.’ is part of the haskell code. Any symbol that is not in bold needs to be written in the program in order to parse correctly.
There are a lot of parts of the grammar that were tricky for me to implement in K. For instance, a sort definition with an optional part could be just written using a ‘|’ in the K syntax. So 
\begin{lstlisting}
qvarid -> [ modid . ] varid
\end{lstlisting}
Is written in my K syntax as
\begin{lstlisting}
syntax QVarId ::= VarId | ModId "." VarId [klabel('qVarIdCon)]
\end{lstlisting}
However, an issue arises when you have something written like
\begin{lstlisting}
data [context =>] simpletype [= constrs] [deriving]
\end{lstlisting}
As an option for the definition of the topdecl sort. What I did was, for each optional part, replace the optional part with a new sort. For instance, I replaced [ context => ] with a new sort called OptContext. Then I just specified
\begin{lstlisting}
syntax OptContext ::= Context "=>" | "" [onlyLabel, klabel('emptyContext)]
\end{lstlisting}
\begin{lstlisting}
// Syntax from haskell 2010 Report
// https://www.haskell.org/onlinereport/haskell2010/haskellch10.html#x17-17500010

module HASKELL-SYNTAX

    syntax Integer ::= Token{([0-9]+)
                | (([0][o]|[0][O])[0-7]+) 
                | (([0][x] | [0][X])[0-9a-fA-F]+)}   [onlyLabel]

    syntax CusFloat ::= Token{([0-9]+[\.][0-9]+([e E][\+\-]?[0-9]+)?)
                                 |([0-9]+[e E][\+\-]?[0-9]+)} [onlyLabel]
    syntax CusChar ::= Token{[\'](~[\'\\\&])[\']} [onlyLabel]
    syntax CusString ::= Token{[\"](~[\"\\]*)[\"]} [onlyLabel]
\end{lstlisting}
\begin{lstlisting}
    syntax VarId ::= Token{[a-z\_][a-z A-Z\_0-9\']*} [onlyLabel] | "size" [onlyLabel]
\end{lstlisting}
I ran into another issue where a program with a variable called ‘size’ did not parse. I found out that this is because ‘size’ is a k keyword. So I just specified that a variable could be a variable token, or “size”.
\begin{lstlisting}
    syntax ConId ::= Token{[A-Z][a-zA-Z \_0-9\']*} [onlyLabel]
    syntax VarSym ::= Token{
   ([\! \# \$ \% \& \* \+ \/ \> \? \^][\! \# \$ \% \& \* \+ \. \/ \< \= \> \? \@ \\ \^ \| \- \~ \:]*)
  |[\-] | [\.]
  |([\.][\! \# \$ \% \& \* \+ \/ \< \= \> \? \@ \\ \^ \| \- \~ \:][\! \# \$ \% \& \* \+ \. \/ \< \= \> \? \@ \\ \^ \| \- \~ \:]*)
  | ([\-][\! \# \$ \% \& \* \+ \. \/ \< \= \? \@ \\ \^ \| \~ \:][\! \# \$ \% \& \* \+ \. \/ \< \= \> \? \@ \\ \^ \| \~ \:]*)
  | ([\@][\! \# \$ \% \& \* \+ \. \/ \< \= \> \? \@ \\ \^ \| \- \~ \:]+)
  | ([\~][\! \# \$ \% \& \* \+ \. \/ \< \= \> \? \@ \\ \^ \| \- \~ \:]+)
  | ([\\][\! \# \$ \% \& \* \+ \. \/ \< \= \> \? \@ \\ \^ \| \- \~ \:]+)
  | ([\|][\! \# \$ \% \& \* \+ \. \/ \< \= \> \? \@ \\ \^ \| \- \~ \:]+)
  | ([\:][\! \# \$ \% \& \* \+ \. \/ \< \= \> \? \@ \\ \^ \| \- \~][\! \# \$ \% \& \* \+ \. \/ \< \= \> \? \@ \\ \^ \| \- \~ \:]*)
  | ([\<][\! \# \$ \% \& \* \+ \. \/ \< \= \> \? \@ \\ \^ \| \~ \:][\! \# \$ \% \& \* \+ \. \/ \< \= \> \? \@ \\ \^ \| \- \~ \:]*)
  | ([\=][\! \# \$ \% \& \* \+ \. \/ \< \= \? \@ \\ \^ \| \~ \:][\! \# \$ \% \& \* \+ \. \/ \< \= \> \? \@ \\ \^ \| \- \~ \:]*)} [onlyLabel]
    syntax ConSym ::= Token{[\:][\! \# \$ \% \& \* \+ \. \/ \< \= \> \? \@ \\ \^ \| \- \~][\! \# \$ \% \& \* \+ \. \/ \< \= \> \? \@ \\ \^ \| \- \~ \:]*}   [onlyLabel]

    syntax IntFloat ::= "(" Integer ")"  [bracket] //NOT OFFICIAL SYNTAX
                      | "(" CusFloat ")" [bracket]
\end{lstlisting}
I ran into an issue where floats and integers did not parse correctly. They caused parsing errors due to ambiguity of parsing. For example the number 123.45 had ambiguity where the parser did not know if 1, 12, or 123 where integers, and if 5 was an integer, or if the entire thing was one float. Normally in K, different tokens are separated with whitespaces. However, for some reason the parser had difficulty here. Initially, I added a workaround by requiring parentheses around each integer and floating point.
f y z (2)
This fixed the issue.
\begin{lstlisting}
    syntax Literal ::= IntFloat | CusChar | CusString
    syntax TyCon ::= ConId
    syntax ModId ::= ConId | ConId "." ModId  [klabel('conModId)]
    syntax QTyCon ::= TyCon | ModId "." TyCon [klabel('conTyCon)]
    syntax QVarId ::= VarId | ModId "." VarId [klabel('qVarIdCon)]
    syntax QVarSym ::= VarSym | ModId "." VarSym  [klabel('qVarSymCon)]
    syntax QConSym ::= ConSym | ModId "." ConSym  [klabel('qConSymCon)]
/*    syntax QTyCls ::= QTyCon
    syntax TyCls ::= ConId
*/
    syntax TyVars ::= List{TyVar, ""} [klabel('typeVars)] //used in SimpleType syntax
    syntax TyVar ::= VarId
    syntax TyVarTuple ::= TyVar "," TyVar      [klabel('twoTypeVarTuple)]
                        | TyVar "," TyVarTuple [klabel('typeVarTupleCon)]

    syntax Con ::= ConId | "(" ConSym ")"   [klabel('conSymBracket)]
    syntax Var ::= VarId | "(" VarSym ")"   [klabel('varSymBracket)]
    syntax QVar ::= QVarId | "(" QConSym ")" [klabel('qVarBracket)]
    syntax QCon ::= QTyCon | "(" GConSym ")"  [klabel('gConBracket)]

    syntax QConOp ::= GConSym | "`" QTyCon "`"    [klabel('qTyConQuote)]
    syntax QVarOp ::= QVarSym | "`" QVarId "`"    [klabel('qVarIdQuote)]
    syntax VarOp ::= VarSym   | "`" VarId "`"     [klabel('varIdQuote)]
    syntax ConOp ::= ConSym   | "`" ConId "`"     [klabel('conIdQuote)]

    syntax GConSym ::= ":" | QConSym
    syntax Vars ::= Var
                  | Var "," Vars  [klabel('varCon)]
    syntax VarsType ::= Vars "::" Type  [klabel('varAssign)]
    syntax Ops ::= Op
                 | Op "," Ops  [klabel('opCon)]
    syntax Fixity ::= "infixl" | "infixr" | "infix"
    syntax Op ::= VarOp | ConOp
    syntax CQName ::= Var | Con | QVar

    /* syntax QConId ::= ConId | ModId "." ConId */

    syntax QOp ::= QVarOp | QConOp
\end{lstlisting}
\begin{lstlisting}

    syntax ModuleName ::= "module" ModId [klabel('moduleName)]

    syntax Module ::= ModuleName "where" Body          [klabel('module)]
                    | ModuleName Exports "where" Body  [klabel('moduleExp)]
                    | Body                             [klabel('moduleBody)]

    syntax Body ::= "{" ImpDecls ";" TopDecls "}" [klabel('bodyimpandtop)]
                  | "{" ImpDecls "}" [klabel('bodyimpdecls)]
                  | "{" TopDecls "}" [klabel('bodytopdecls)]

    syntax ImpDecls ::= List{ImpDecl, ";"} [klabel('impDecls)]
\end{lstlisting}

The sort that contains all the other sorts is a module. A module represents one complete Haskell program. It can have either a name and a body, a name and a body with exports, or just a body.

\begin{lstlisting}
module Foo where
{import Bar
}
\end{lstlisting}

This example program has a Module with only ImpDecls. It has one ImpDecl.

\begin{lstlisting}
import Bar
\end{lstlisting}

This example program has a Module with no name and only ImpDecls. It has one ImpDecl.

\begin{lstlisting}
    syntax Exports ::= "(" ExportList OptComma ")"
    syntax ExportList ::= List{Export, ","}

    syntax Export ::= QVar 
                    | QTyCon OptCQList
                    | ModuleName

    //optional cname list
    syntax OptCQList ::= "(..)"
                       | "(" CQList ")"   [klabel('cqListBracket)]
                          //Liyi: a check needs to place in preprocessing to check
                          //if the CQList is a cname list or a qvar list.
                       | "" [onlyLabel, klabel('emptyOptCNameList)]
    syntax CQList ::= List{CQName, ","}

    syntax ImpDecl ::= "import" OptQualified ModId OptAsModId OptImpSpec [klabel('impDecl)]
                     | "" [onlyLabel, klabel('emptyImpDecl)]
    syntax OptQualified ::= "qualified"
                          | ""  [onlyLabel, klabel('emptyQualified)]
    syntax OptAsModId ::= "as" ModId
                        | ""    [onlyLabel, klabel('emptyOptAsModId)]

    syntax OptImpSpec ::= ImpSpec
                        | ""    [onlyLabel, klabel('emptyOptImpSpec)]

    syntax ImpSpecKey ::= "(" ImportList OptComma ")"
    syntax ImpSpec ::= ImpSpecKey
                     | "hiding" ImpSpecKey

    syntax ImportList ::= List{Import, ","}

    syntax Import ::= Var
                    | TyCon CQList
\end{lstlisting}

\begin{lstlisting}
    syntax TopDecls ::= List{TopDecl, ";"} [klabel('topdeclslist)]

    syntax TopDecl ::= Decl [klabel('topdecldecl)]
                     > "type" SimpleType "=" Type [klabel('type)]
                     | "data" OptContext SimpleType OptConstrs OptDeriving  [klabel('data)]
                     | "newtype" OptContext SimpleType "=" NewConstr OptDeriving [klabel('newtype)]
                     | "class" OptContext ConId TyVar OptCDecls [klabel('class)]
                     | "instance" OptContext QTyCon Inst OptIDecls [klabel('instance)]
                     | "default" Types [klabel('default)]
                     | "foreign" FDecl [klabel('foreign)]
\end{lstlisting}

The main types of expressions in Haskell are TopDecls - Top Declarations. A top declaration can either be a type, a data, a newtype, a class, an instance, a default, a foreign, or an arbitrary declaration.

\begin{lstlisting}
    syntax FDecl ::= "import" CallConv CusString Var "::" FType
                   | "import" CallConv Safety CusString Var "::" FType
                   | "export" CallConv Safety CusString Var "::" FType
    //Liyi: fdecl needs to use special function in preprocessing
    // to get the actually elements from the impent and expent from the CusString
    //did string analysis
    
    syntax Safety ::= "unsafe" | "safe"
                   
    syntax CallConv ::= "ccall" | "stdcall" | "cplusplus" | "jvm" | "dotnet"
    syntax FType ::= FrType
                   | FaType "->" FType // unsure about this one syntax is ambiguous UNFINISHED
    
    syntax FrType ::= FaType
                    | "()"
                    
    syntax FaType ::= QTyCon ATypeList

    //define declaration.
    syntax OptDecls ::= "where" Decls | "" [onlyLabel, klabel('emptyOptDecls)]
    syntax Decls ::= "{" DeclsList "}" [klabel('decls)]
    syntax DeclsList ::= List{Decl, ";"} [klabel('declsList)]

    syntax Decl ::= GenDecl
                  | FunLhs Rhs [klabel('declFunLhsRhs)]
                  | Pat Rhs [klabel('declPatRhs)]

    syntax OptCDecls ::= "where" CDecls | "" [onlyLabel, klabel('emptyOptCDecls)]
    syntax CDecls ::= "{" CDeclsList "}"
    syntax CDeclsList ::= List{CDecl, ";"}

    syntax CDecl ::= GenDecl
                   | FunLhs Rhs
                   | Var Rhs

    syntax OptIDecls ::= "where" IDecls | "" [onlyLabel, klabel('emptyOptIDecls)]
    syntax IDecls ::= "{" IDeclsList "}"
    syntax IDeclsList ::= List{IDecl, ";"} [klabel('ideclslist)]

    syntax IDecl ::= FunLhs Rhs [klabel('cdeclFunLhsRhs)]
                   | Var Rhs [klabel('cdeclVarRhs)]
                   | "" [onlyLabel, klabel('emptyIDecl)]

    syntax GenDecl ::= VarsType
                     | Vars "::" Context "=>" Type   [klabel('genAssignContext)]
                     | Fixity Ops
                     | Fixity Integer Ops
                     | "" [onlyLabel, klabel('emptyGenDecl)]

    //three optional data type for the TopDecl data operator. 
    //deriving data type
    syntax OptDeriving ::= Deriving | "" [onlyLabel, klabel('emptyDeriving)]
    syntax Deriving ::= "deriving" DClass
                      | "deriving" "(" DClassList ")"
    syntax DClassList ::= List{DClass, ","}
    syntax DClass ::= QTyCon

    syntax FunLhs ::= Var APatList [klabel('varApatList)]
                    | Pat VarOp Pat [klabel('patVarOpPat)]
                    | "(" FunLhs ")" APatList [klabel('funlhsApatList)]

    syntax Rhs ::= "=" Exp OptDecls [klabel('eqExpOptDecls)]
                 | GdRhs OptDecls [klabel('gdRhsOptDecls)]

    syntax GdRhs ::= Guards "=" Exp
                   | Guards "=" Exp GdRhs
    syntax Guards ::= "|" GuardList
    syntax GuardList ::= Guard | Guard "," GuardList  [klabel('guardListCon)]
    syntax Guard ::= Pat "<-" InfixExp
                   | "let" Decls
                   | InfixExp

    //definition of exp
    syntax Exp ::= InfixExp
                 > InfixExp "::" Type  [klabel('expAssign)]
                 | InfixExp "::" Context "=>" Type  [klabel('expAssignContext)]

    syntax InfixExp ::= LExp
                      > "-" InfixExp   [klabel('minusInfix)]
                      > LExp QOp InfixExp

    syntax LExp ::= AExp
                  > "\\" APatList "->" Exp [klabel('lambdaFun)]
                  | "let" Decls "in" Exp [klabel('letIn)]
                  | "if" Exp OptSemicolon "then" Exp OptSemicolon "else" Exp [klabel('ifThenElse)]
                  | "case" Exp "of" "{" Alts "}" [klabel('caseOf)]
                  | "do" "{" Stmts "}" [klabel('doBlock)]
                  //| FExp

    //syntax FExp ::= AExp | FExp AExp

    syntax OptSemicolon ::= ";" | "" [onlyLabel, klabel('emptySemicolon)]
    syntax OptComma ::= "," | ""     [onlyLabel, klabel('emptyComma)]

//    syntax OptWhereDecls ::= "where" Decls | ""

    syntax AExp ::= QVar [klabel('aexpQVar)]
                  | GCon [klabel('aexpGCon)]
                  | Literal [klabel('aexpLiteral)]
                  > AExp AExp [left, klabel('funApp)]
                  > QCon "{" FBindList "}"
                  | AExp "{" FBindList "}" //aexp cannot be qcon UNFINISHED
                          //Liyi: first, does not understand the syntax, it is the Qcon {FBindlist}
                          //or QCon? Second, place a check in preprosssing.
                          //and also check the Fbindlist here must be at least one argument
                  > "(" Exp ")"             [bracket]
                  | "(" ExpTuple ")"
                  | "[" ExpList "]"
                  | "[" Exp OptExpComma ".." OptExp "]"
                  | "[" Exp "|" Quals "]"
                  | "(" InfixExp QOp ")"
                  | "(" QOp InfixExp ")" //qop cannot be - (minus) UNFINISHED
                           //Liyi: place a check here to check if QOp is a minus

    syntax OptExpComma ::= "," Exp | "" [onlyLabel, klabel('emptyExpComma)]
    syntax OptExp ::= Exp | "" [onlyLabel, klabel('emptyExp)]

    syntax ExpList ::= Exp | Exp "," ExpList  [right]
    syntax ExpTuple ::= Exp "," Exp           [right, klabel('twoExpTuple)]
                      | Exp "," ExpTuple      [right, klabel('expTupleCon)]

    //constr datatypes
    syntax OptConstrs  ::= "=" Constrs [klabel('nonemptyConstrs)] | "" [onlyLabel, klabel('emptyConstrs)]
    syntax Constrs     ::= Constr [klabel('singleConstr)] | Constr "|" Constrs [klabel('multConstr)]
    syntax Constr      ::= Con OptBangATypes [klabel('constrCon)] // (arity con  =  k, k ≥ 0) UNFINISHED
                         | SubConstr ConOp SubConstr
                         | Con "{" FieldDeclList "}"

    syntax NewConstr   ::= Con AType [klabel('newConstrCon)]
                         | Con "{" Var "::" Type "}"

    syntax SubConstr   ::= BType | "!" AType
    syntax FieldDeclList ::= List{FieldDecl, ","}
    syntax FieldDecl ::= VarsType
                       | Vars "::" "!" AType


    syntax OptBangATypes ::= List{OptBangAType, " "} [klabel('optBangATypes)]
    syntax OptBangAType ::= OptBang AType [klabel('optBangAType)]
    syntax OptBang ::= "!" | "" [onlyLabel, klabel('emptyBang)]

    syntax OptContext ::= Context "=>" | "" [onlyLabel, klabel('emptyContext)]
    syntax Context ::= Class
                     | "(" Classes ")"

    syntax Classes ::= List{Class, ","}

    syntax SimpleClass ::= QTyCon TyVar  [klabel('classCon)]

    syntax Class       ::= SimpleClass
                         | QTyCon "(" TyVar ATypeList ")"
                              //Liyi: a check in preprossing to check if the Atype list is empty
                              //it must have at least one item

    //define type and simple type
    syntax SimpleType  ::= TyCon TyVars  [klabel('simpleTypeCon)]
    syntax Type ::= BType
                  | BType "->" Type  [klabel('typeArrow)]
    syntax BType ::= AType
                   | BType AType [klabel('baTypeCon)]

    syntax ATypeList ::= List{AType, ""} [klabel('atypeList)]

    syntax AType ::= GTyCon                     [klabel('atypeGTyCon)]
                   | TyVar                      [klabel('atypeTyVar)]
                   | "(" TypeTuple ")"          [klabel('atypeTuple)]
                   | "[" Type "]"               [klabel('tyList)]
                   | "(" Type ")"               [bracket]
    syntax TypeTuple ::= Type "," Type          [right,klabel('twoTypeTuple)]
                       | Type "," TypeTuple     [klabel('typeTupleCon)]
    syntax Types ::= List{Type, ","}

    syntax GConCommas ::= "," | "," GConCommas
    syntax GConCommon ::= "()" | "[]" | "(" GConCommas ")" //was incorrect syntax
    syntax GTyCon ::= QTyCon
                    | GConCommon
                    | "(->)"

    syntax GCon ::= GConCommon
                  | QCon

    //inst definition
    syntax Inst  ::= GTyCon
                   | "(" GTyCon TyVars ")" //TyVars must be distinct UNFINISHED
                   | "(" TyVarTuple ")" //TyVars must be distinct
                   | "[" TyVar "]"  [klabel('tyVarList)]
                   | "(" TyVar "->" TyVar ")" //TyVars must be distinct
    //pat definition
    syntax Pat ::= LPat QConOp Pat
                 | LPat

    syntax LPat ::= APat
                  | "-" IntFloat    [klabel('minusPat)]
                  | GCon APatList  [klabel('lpatCon)]//arity gcon = k UNFINISHED

    syntax APatList ::= APat | APat APatList [klabel('apatCon)]

    syntax APat ::= Var [klabel('apatVar)]
                  | Var "@" APat
                  | GCon
                  | QCon "{" FPats "}"
                  | Literal [klabel('apatLiteral)]
                  | "_"
                  | "(" Pat ")"   [bracket]
                  | "(" PatTuple ")"
                  | "[" PatList "]"
                  | "~" APat

    syntax PatTuple ::= Pat "," Pat         [klabel('twoPatTuple)]
                      | Pat "," PatTuple    [klabel('patTupleCon)]
    syntax PatList ::= Pat
                      | Pat "," PatList     [klabel('patListCon)]

    syntax FPats ::= List{FPat, ","}
    syntax FPat ::= QVar "=" Pat

    //definition of quals
    syntax Quals ::= Qual | Qual "," Quals  [klabel('qualCon)]

    syntax Qual ::= Pat "<-" Exp
                  | "let" Decls
                  | Exp

    //definition of alts
    syntax Alts ::= Alt | Alt ";" Alts

    syntax Alt ::= Pat "->" Exp  [klabel('altArrow)]
                 | Pat "->" Exp "where" Decls
                 | "" [onlyLabel, klabel('emptyAlt)]

    //definition of stmts
    syntax Stmts ::= StmtList Exp OptSemicolon
    syntax StmtList ::= List{Stmt, ""}
    syntax Stmt ::= Exp ";"
                  | Pat "<-" Exp ";"
                  | "let" Decls ";"
                  | ";"

     //definition of fbind
    syntax FBindList ::= List{FBind, ","}
    syntax FBind ::= QVar "=" Exp
\end{lstlisting}