The next step is the actual type inferencing algorithm.
	I needed to create a syntax for polymorphic types that may contain monomorphic type variables and polymorphic type variables.
	Then I made a map from type constructor names to arities, called Delta.
	Then I made a map from data constructors and term identifiers to their most general polymorphic types, called beta.
	The first part was converting the T data structure into beta, which is more suited for type inferencing.
	Something to note is that for GHC:
f x = y x
y x = f x
This set of functions is allowed. When run, the function just simply runs forever.
	Another thing to note is that the
[context =>]
part of the syntax for the types is deprecated.
https://stackoverflow.com/questions/9345589/guards-vs-if-then-else-vs-cases-in-haskell
For functions, function guards, cases, and if-then-else are all equivalent.