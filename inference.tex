\chapter{Inferencing}
\section{Data Structures}
The next step is the actual type inferencing algorithm.
	I needed to create a syntax for polymorphic types that may contain monomorphic type variables and polymorphic type variables.
	Then I made a map from type constructor names to arities, called Delta.
	Then I made a map from data constructors and term identifiers to their most general polymorphic types, called beta.
	The first part was converting the T data structure into beta, which is more suited for type inferencing.
	Something to note is that for GHC:
	$$
f x = y x
$$
$$
y x = f x
$$
This set of functions is allowed. When run, the function just simply runs forever.
	Another thing to note is that the
\begin{lstlisting}
[context =>]
\end{lstlisting}
part of the syntax for the types is deprecated.
https://stackoverflow.com/questions/9345589/guards-vs-if-then-else-vs-cases-in-haskell
For functions, function guards, cases, and if-then-else are all equivalent.

A polymorphic data type looks like $\forall a b c, (a \rightarrow b) \rightarrow c$

\subsection{Inferencing Rules}

\begin{prooftree}
\AxiomC{Hyp}
\UnaryInfC{$\{A \land B\} \vdash A \land B$}
\AxiomC{Hyp}
\UnaryInfC{$\{A \land B\} \cup \{B\} \vdash B$}
\RightLabel{And$_R$ E}
\BinaryInfC{$\{A \land B\} \vdash B$}
\AxiomC{Hyp}
\UnaryInfC{$\{A \land B\} \vdash A \land B$}
\AxiomC{Hyp}
\UnaryInfC{$\{A \land B\} \cup \{A\} \vdash A$}
\RightLabel{And$_L$ E}
\BinaryInfC{$\{A \land B\} \vdash A$}
\RightLabel{And I}
\BinaryInfC{$\{A \land B\} \vdash B \land A$}
\RightLabel{Imp I}
\UnaryInfC{$\{\} \vdash A \land B \Rightarrow B \land A$}
\end{prooftree}

Haskell is a strong and static type system.

This means that type inferencing can be ran before compilation or running the code. used to ensure that fun

Type inferencing 

Haskell's type system is a Hindley-Milner polymorphic type system that has been extended with type classes to account for overloaded function
%[haskell 2010 report] 

A type system is a set of rules that assign a property to various constructs in a programming language called type. A type is a property that allows the programmer to add constraints to programs.
%[https://en.wikipedia.org/wiki/Type_system]

\section{Lambda Calculus}

\section{Type theory}
Type theory was created by Bertrand Russell to prevent Russell's Paradox for set theory, introduced by Georg Cantor. The issue was that not specifying a certain property for sets allowed sets to contain themselves in Naive Set Theory. So Bertrand Russell prevented this problem by specifying a property called type for objects, and objects cannot contain their own type.
%[https://plato.stanford.edu/entries/type-theory/]

\section{Hindley-Milner}

\section{Inferencing Algorithm}

\begin{prooftree}
\AxiomC{}
\RightLabel{Constant}
\UnaryInfC{$\Gamma \vdash c : \tau \, | \, \text{unify}\{ ( \tau , \text{freshInstance}(\tau))\}$}
\end{prooftree}

\begin{prooftree}
\AxiomC{}
\RightLabel{Variable}
\UnaryInfC{$\Gamma \vdash x : \tau \, | \, \text{unify}\{ ( \tau , \text{freshInstance}(\Gamma(x)))\}$}
\end{prooftree}

\begin{prooftree}
\AxiomC{$[x : \tau_1] + \Gamma \vdash e : \tau_2 \, | \, \sigma$}
\RightLabel{Lambda}
\UnaryInfC{$\Gamma \vdash \text{\textbackslash \, x \textrightarrow \, e}: \tau \, | \, \text{unify}\{ ( \sigma(\tau) , \sigma(\tau_1 \rightarrow \tau_2)) \} \circ \sigma$}
\end{prooftree}

\begin{prooftree}
\AxiomC{$\Gamma \vdash e_1 : \text{bool} \, | \, \sigma_1$}
\AxiomC{$\sigma_1(\Gamma) \vdash e_2 : \sigma_1(\tau) \, | \, \sigma_2$}
\AxiomC{$\sigma_2 \circ \sigma_1(\Gamma) \vdash e_3 : \sigma_2 \circ \sigma_1(\tau) \, | \, \sigma_3$}
\RightLabel{IfThenElse}
\TrinaryInfC{$\Gamma \vdash \text{if} \, e_1 \, \text{then} \, e_2 \, \text{else} \, e_3 : \tau \, | \, \sigma_3 \circ \sigma_2 \circ \sigma_1$}
\end{prooftree}

\begin{prooftree}
\AxiomC{$\Gamma \vdash e_1 : \tau_1 \rightarrow \tau \, | \, \sigma_1$}
\AxiomC{$\sigma_1(\Gamma) \vdash e_2 : \sigma_1(\tau_1) \, | \, \sigma_2$}
\RightLabel{Application}
\BinaryInfC{$\Gamma \vdash e_1 e_2 : \tau \, | \, \sigma_2 \circ \sigma_1$}
\end{prooftree}

\begin{prooftree}
\AxiomC{}
\RightLabel{LetIn}
\UnaryInfC{$\Gamma \vdash c : \tau \, | \, \text{unify}\{ ( \tau , \text{freshInstance}(\tau))\}$}
\end{prooftree}

[cs 421 gather exp ty substitution mp]

\subsection{Unification Algorithm}

Let $S = \{(s_1 , t_1), (s_2 , t_2), \cdots, (s_n , t_n)\}$ be a
unification problem.

Case $S = \{ \} : \text{Unif}(S) = \text{Identity function}$ (ie
no substitution)

Case $S = {(s, t)} \cup S')$: Four main steps

Delete: if $s = t$ (they are the same term)
then $Unif(S) = Unif(S')$

Decompose: if $s = f(q_1, \cdots, q_m )$ and
$t = f(r_1, \cdots, r_m )$ (same f, same m!), then
$\text{Unif}(S) = \text{Unif}(\{(q_1 , r_1 ), ..., (q_m , r_m )\} \cup S')$

Orient: if $t = x$ is a variable, and s is not a
variable, $\text{Unif}(S) = \text{Unif} (\{(x,s)\} \cup S')$

Eliminate: if s = x is a variable, and
x does not occur in t (the occurs
check), then

Let $\phi = x \mapsto t$

Let $\psi = \text{Unif}(\phi(S'))$

$\text{Unif}(S) = \{x \mapsto \psi(t) \} \circ \psi$

Note: {x |→ a} o {y |→ b} =
{y |→ ({x |→ a}(b)} o {x |→ a} if y not
in a

[cs 421 class notes]
