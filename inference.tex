\chapter{Inferencing}
\section{Data Structures}
The next step is the actual type inference algorithm.
	I needed to create a syntax for polymorphic types that may contain monomorphic type variables and polymorphic type variables.
	Then I made a map from type constructor names to arities, called Delta.
	Then I made a map from data constructors and term identifiers to their most general polymorphic types, called beta.
	The first part was converting the T data structure into beta, which is more suited for type inferencing.
\subsection{Transform T into Beta}
\begin{lstlisting}
    // STEP 3 Transform T into beta

    syntax KItem ::= "startTTransform"
    syntax KItem ::= "constructDelta"
    syntax KItem ::= "constructBeta"

    rule <k> startTTransform
             => constructDelta
                ~> (constructBeta) ...</k>

    rule <k> constructDelta
             => makeDelta(T,.Map) ...</k>
         <tempT> T:K </tempT>

    syntax KItem ::= makeDelta(K,Map) [function] //(T,Delta)
    syntax KItem ::= newDelta(Map) //Delta
    syntax KItem ::= newBeta(Map) //beta
    syntax List ::= retPolyList(K,List) [function] //(T,Delta)
\end{lstlisting}

\subsubsection{Construct Delta}
\begin{lstlisting}
    rule makeDelta(TList(ListItem(TObject(ModName:K,A:K,Polys:K,C:K)) Rest:List),M:Map) =>
         makeDelta(TList(Rest),M[ModPlusType(ModName,A) <- size(retPolyList(Polys,.List))])
    rule makeDelta(TList(.List),M:Map) => newDelta(M)

    rule retPolyList('typeVars(A:K,,Rest:K),NewList:List) => retPolyList(Rest, ListItem(A) NewList)
    rule retPolyList(.TyVars,L:List) => L

    rule <k> newDelta(M:Map)
             => .K ...</k>
         <tempDelta> OldDelta:K => M </tempDelta>

\end{lstlisting}

\subsubsection{Construct Beta}
\begin{lstlisting}
    rule <k> constructBeta
             => makeBeta(T,.Map) ...</k>
         <tempT> T:K </tempT>

    syntax KItem ::= makeBeta(K,Map) [function] //(T,Beta,Delta)

    rule makeBeta(TList(ListItem(TObject(ModName:K,A:K,B:K,ListItem(InnerTPiece(Con:K,H:K,D:K,E:K,F:K)) InnerRest:List)) Rest:List),Beta:Map) =>
         makeBeta(TList(ListItem(TObject(ModName,A,B,InnerRest)) Rest),Beta[ModPlusType(ModName,Con) <- betaParser(E,B,A)])
    rule makeBeta(TList(ListItem(TObject(ModName:K,A:K,B:K,.List)) Rest:List),Beta:Map) =>
         makeBeta(TList(Rest),Beta)
    rule makeBeta(TList(.List),Beta:Map) =>
         newBeta(Beta)
//    rule makeBeta(TList(ListItem(TObject(ModName:K,A:K,B:K,ListItem(InnerTPiece(C:K,H:K,D:K,E:K,F:K)) InnerRest:List)) Rest:List),Beta:Map) =>
//         makeBeta(TList(ListItem(TObject(ModName,A,B,InnerRest)) Rest),Beta)

    syntax KItem ::= betaParser(K,K,K) [function] //(Tree Piece,NewSyntax,Parameters,Constr)
    syntax Set ::= getTyVarsRHS(K,List) [function]

    syntax KItem ::= forAll(Set,K)
    syntax KItem ::= funtype(K,K)

    syntax Set ::= listToSet(List, Set) [function]

    rule listToSet(ListItem(A:KItem) L:List, S:Set) => listToSet(L, SetItem(A) S)
    rule listToSet(.List, S:Set) => S


//if optbangAtypes, need to see if first variable is a typecon
//if its a typecon then need to go into Delta and see the amount of parameters it has
//then count the number of optbangAtypes after the typecon
    rule betaParser('constrCon(A:K,, B:K), Par:K, Con:K) => forAll(getTyVarsRHS(B,.List), betaParser(B, Par, Con))
    rule betaParser('optBangATypes('optBangAType('emptyBang(.KList),, 'atypeTyVar(Tyv:K)),, Rest:K), Par:K, Con:K) => funtype(Tyv, betaParser(Rest, Par, Con))
    rule betaParser('optBangATypes('optBangAType('emptyBang(.KList),, 'baTypeCon(A:K,, B:K)),, Rest:K), Par:K, Con:K) => funtype('baTypeCon(A:K,, B:K), betaParser(Rest, Par, Con))
    rule betaParser('optBangATypes('optBangAType('emptyBang(.KList),, 'atypeGTyCon(Tyc:K)),, Rest:K), Par:K, Con:K) => funtype(Tyc, betaParser(Rest, Par, Con))
    rule betaParser(.OptBangATypes, Par:K, Con:K) => 'simpleTypeCon(Con,, Par)
//    rule betaParser('optBangATypes('optBangAType('emptyBang(.KList),, 'atypeGTyCon(Tyc:K)),, Rest:KItem)) => getTypeVars(A) getTypeVars(Rest)
//    rule getTypeVars('optBangAType('emptyBang(.KList),, Rest:K)) => getTypeVars(Rest)
//    rule getTypeVars('atypeGTyCon(A:K)) => .List
//    rule getTypeVars('atypeTyVar(A:K)) => ListItem(A)
//    rule getTypeVars(.OptBangATypes) => .List

    rule getTyVarsRHS(.OptBangATypes,Tylist:List) => listToSet(Tylist, .Set)

    rule <k> newBeta(M:Map)
             => .K ...</k>
         <tempBeta> OldBeta:K => M </tempBeta>
\end{lstlisting}
	Something to note is that mutually recursive functions are allowed in Haskell.
	For example:
	$$
f x = y x
$$
$$
y x = f x
$$
This set of functions is allowed to compile. When run, the function just simply runs forever. This is unlike the OCaml semantics, which does not allow for mutually recursive functions.

	Another thing to note is that the
\begin{lstlisting}
[context =>]
\end{lstlisting}
part of the syntax for the types is deprecated.
https://stackoverflow.com/questions/9345589/guards-vs-if-then-else-vs-cases-in-haskell
For functions, function guards, cases, and if-then-else are all equivalent.

A polymorphic data type looks like $\forall a b c, (a \rightarrow b) \rightarrow c$

\section{Inferencing Rules}

Haskell is a strong and static type system.

This means that type inferencing can be ran before compilation or running the code. used to ensure that fun

Type inferencing 

Haskell's type system is a Hindley-Milner polymorphic type system that has been extended with type classes to account for overloaded function
%[haskell 2010 report] 

A type system is a set of rules that assign a property to various constructs in a programming language called type. A type is a property that allows the programmer to add constraints to programs.
%[https://en.wikipedia.org/wiki/Type_system]

\section{Type theory}
Type theory was created by Bertrand Russell to prevent Russell's Paradox for set theory, introduced by Georg Cantor. The issue was that not specifying a certain property for sets allowed sets to contain themselves in Naive Set Theory. So Bertrand Russell prevented this problem by specifying a property called type for objects, and objects cannot contain their own type.
%[https://plato.stanford.edu/entries/type-theory/]

\section{Lambda Calculus}
The Lambda Calculus was created to

\section{Hindley-Milner}

\section{Definition of Substitution}
A substitution is a set of variables and their replacements. Applying a substitution to an expression means to simultaneously replace each variable in the expression with the replacement term.

[http://www.mathcs.duq.edu/simon/Fall04/notes-7-4/node3.html]

\section{Composition of Substitutions}

\section{Inferencing Algorithm}

\begin{prooftree}
\AxiomC{}
\RightLabel{Constant}
\UnaryInfC{$\Gamma \vdash c : \tau \, | \, \text{unify}\{ ( \tau , \text{freshInstance}(\tau))\}$}
\end{prooftree}

\begin{prooftree}
\AxiomC{}
\RightLabel{Variable}
\UnaryInfC{$\Gamma \vdash x : \tau \, | \, \text{unify}\{ ( \tau , \text{freshInstance}(\Gamma(x)))\}$}
\end{prooftree}

\begin{prooftree}
\AxiomC{$[x : \tau_1] + \Gamma \vdash e : \tau_2 \, | \, \sigma$}
\RightLabel{Lambda}
\UnaryInfC{$\Gamma \vdash \text{\textbackslash \, x \textrightarrow \, e}: \tau \, | \, \text{unify}\{ ( \sigma(\tau) , \sigma(\tau_1 \rightarrow \tau_2)) \} \circ \sigma$}
\end{prooftree}

\begin{prooftree}
\AxiomC{$\Gamma \vdash e_1 : \text{bool} \, | \, \sigma_1$}
\AxiomC{$\sigma_1(\Gamma) \vdash e_2 : \sigma_1(\tau) \, | \, \sigma_2$}
\AxiomC{$\sigma_2 \circ \sigma_1(\Gamma) \vdash e_3 : \sigma_2 \circ \sigma_1(\tau) \, | \, \sigma_3$}
\RightLabel{IfThenElse}
\TrinaryInfC{$\Gamma \vdash \text{if} \, e_1 \, \text{then} \, e_2 \, \text{else} \, e_3 : \tau \, | \, \sigma_3 \circ \sigma_2 \circ \sigma_1$}
\end{prooftree}

\begin{prooftree}
\AxiomC{$\Gamma \vdash e_1 : \tau_1 \rightarrow \tau \, | \, \sigma_1$}
\AxiomC{$\sigma_1(\Gamma) \vdash e_2 : \sigma_1(\tau_1) \, | \, \sigma_2$}
\RightLabel{Application}
\BinaryInfC{$\Gamma \vdash e_1 e_2 : \tau \, | \, \sigma_2 \circ \sigma_1$}
\end{prooftree}

\begin{prooftree}
\AxiomC{}
\RightLabel{LetIn}
\UnaryInfC{$\Gamma \vdash c : \tau \, | \, \text{unify}\{ ( \tau , \text{freshInstance}(\tau))\}$}
\end{prooftree}

[cs 421 gather exp ty substitution mp]

\begin{lstlisting}
requires "haskell-syntax.k"
requires "haskell-configuration.k"
requires "haskell-preprocessing.k"

module HASKELL-TYPE-INFERENCING
    imports HASKELL-SYNTAX
    imports HASKELL-CONFIGURATION
    imports HASKELL-PREPROCESSING

    syntax KItem ::= "Bool" //Boolean

    // STEP 4 Type Inferencing
    syntax KItem ::= inferenceShell(K) [function]//Input, AlphaMap, Beta, Delta, Gamma
    //syntax KItem ::= typeInferenceFun(K,Map,Map,Map,Map,K,K) [function]//Input, Alpha, Beta, Delta, Gamma
    //syntax KItem ::= typeInferenceFun(Map,K,K) //Gamma, Expression, Guessed Type
    syntax Map ::= genGamma(K,Map,K) [function] //Apatlist, Gamma Type
    syntax KItem ::= genLambda(K,K) [function]
    syntax KItem ::= guessType(Int)
//    syntax KItem ::= lambdaReturn(K,K,K)
    syntax KItem ::= freshInstance(K, Int) [function]
    syntax Int ::= paramSize(K) [function]


    syntax KItem ::= mapBag(Map)
    syntax KResult ::= mapBagResult(Map)

    syntax Map ::= gammaSub(Map,Map,Map) [function]//substitution, gamma

    rule <k> performIndividualInferencing => inferenceShell(Code) ...</k>
         <tempModule> Mod:KItem </tempModule>

         <moduleName> 'moduleName(Mod) </moduleName>
         <moduleTempCode> Code:KItem </moduleTempCode>

    rule inferenceShell('topdeclslist('type(A:K,, B:K),, Rest:K)) =>
         inferenceShell(Rest) //constructalpha
    rule inferenceShell('topdeclslist('data(A:K,, B:K,, C:K,, D:K),, Rest:K)) =>
         inferenceShell(Rest)
    rule inferenceShell('topdeclslist('newtype(A:K,, B:K,, C:K,, D:K),, Rest:K)) =>
         inferenceShell(Rest)
    rule inferenceShell('topdeclslist('class(A:K,, B:K,, C:K,, D:K),, Rest:K)) =>
         inferenceShell(Rest)
    rule inferenceShell('topdeclslist('instance(A:K,, B:K,, C:K,, D:K),, Rest:K)) =>
         inferenceShell(Rest)
    rule inferenceShell('topdeclslist('default(A:K,, B:K,, C:K,, D:K),, Rest:K)) =>
         inferenceShell(Rest)
    rule inferenceShell('topdeclslist('foreign(A:K,, B:K,, C:K,, D:K),, Rest:K)) =>
         inferenceShell(Rest)

    rule inferenceShell('topdeclslist('topdecldecl(A:K),, Rest:K)) =>
         typeInferenceFun(.ElemList, .Map,A,guessType(0)) ~> inferenceShell(Rest)


    rule <k> typeInferenceFun(.ElemList, Gamma:Map, 'declFunLhsRhs(Fn:K,, Lhsrhs:K), Guess:K) =>
         typeInferenceFun(.ElemList, Gamma, Lhsrhs, Guess) ...</k>
    rule <k> typeInferenceFun(.ElemList, Gamma:Map, 'eqExpOptDecls(Ex:K,, Optdecls:K), Guess:K) =>
         typeInferenceFun(.ElemList, Gamma, Ex, Guess) ...</k>
\end{lstlisting}

\subsection{Variable Rule}

\begin{prooftree}
\AxiomC{}
\RightLabel{Variable}
\UnaryInfC{$\Gamma \vdash x : \tau \, | \, \text{unify}\{ ( \tau , \text{freshInstance}(\Gamma(x)))\}$}
\end{prooftree}

\begin{lstlisting}
    //T-App
    //rule typeInferenceFun('aexpQVar(Var:K), Alpha:Map, Beta:Map, Delta:Map, (Var |-> Sigma:K) Gamma:Map,.K,.K) => Sigma
    //Gamma Proves x:phi(tau) if Gamma(x) = \forall alpha_1, ..., alpha_n . tau
    //where phi replaces all occurrences of alpha_1, ..., alpha_n by monotypes tau_1, ..., tau_n

    rule <k> typeInferenceFun(.ElemList, (Var |-> Type:K) Gamma:Map, 'aexpQVar(Var:K), Guess:KItem)
          => mapBagResult(uniFun(ListItem(uniPair(Guess,freshInstance(Type, TypeIt))))) ...</k> //Variable rule
         <typeIterator> TypeIt:Int => TypeIt +Int paramSize(Type) </typeIterator>
\end{lstlisting}
\subsection{Constant Rule}

\begin{prooftree}
\AxiomC{}
\RightLabel{Constant}
\UnaryInfC{$\Gamma \vdash c : \tau \, | \, \text{unify}\{ ( \tau , \text{freshInstance}(\tau))\}$}
\end{prooftree}

\begin{lstlisting}

    rule <k> typeInferenceFun(.ElemList, Gamma:Map, 'aexpGCon('conTyCon(Mid:K,, Gcon:K)), Guess:KItem)
          => mapBagResult(uniFun(ListItem(uniPair(Guess,freshInstance(Type, TypeIt))))) ...</k> //Constant rule
         <tempBeta> (ModPlusType(Mid,Gcon) |-> Type:K) Beta:Map </tempBeta>
         <typeIterator> TypeIt:Int => TypeIt +Int paramSize(Type) </typeIterator>
         
\end{lstlisting}

\subsection{Lambda Rule}

\begin{prooftree}
\AxiomC{$[x : \tau_1] + \Gamma \vdash e : \tau_2 \, | \, \sigma$}
\RightLabel{Lambda}
\UnaryInfC{$\Gamma \vdash \text{\textbackslash \, x \textrightarrow \, e}: \tau \, | \, \text{unify}\{ ( \sigma(\tau) , \sigma(\tau_1 \rightarrow \tau_2)) \} \circ \sigma$}
\end{prooftree}

\begin{lstlisting}

   syntax KItem ::= typeInferenceFun(ElemList, Map, K, K) [strict(1)]
   syntax KItem ::= typeInferenceFunLambda(ElemList, K, K, K) [strict(1)]
/* automatically generated by the strict(1) in typeInferenceFun or typeInferenceFunAux
   rule typeInferenceFunAux(Es:ElemList, C:K, A:K, B:K) => Es ~> typeInferenceFun(HOLE, C, A, B)
        requires notBool isKResult(Es)
   rule Es:KResult ~> typeInferenceFunAux(HOLE, C:K,A:K, B:K) => typeInferenceFun(Es, C, A, B)
*/     

    //lambda rule
    rule <k> typeInferenceFun(.ElemList, Gamma:Map, 'lambdaFun(Apatlist:K,, Ex:K), Guess:KItem)
          => typeInferenceFunLambda(val(typeInferenceFun(.ElemList, genGamma(Apatlist,Gamma,guessType(TypeIt)), genLambda(Apatlist,Ex), guessType(TypeIt +Int 1))), .ElemList, Guess, guessType(TypeIt),guessType(TypeIt +Int 1)) ...</k>
         <typeIterator> TypeIt:Int => TypeIt +Int 2 </typeIterator>

    rule <k> typeInferenceFunLambda(valValue(mapBagResult(Sigma:Map)), .ElemList, Tau:K, Tauone:K, Tautwo:K)
         => mapBagResult(compose(uniFun(ListItem(uniPair(typeSub(Sigma,Tau),typeSub(Sigma,funtype(Tauone,Tautwo))))),Sigma)) ...</k>
    
\end{lstlisting}

\subsection{Application Rule}

\begin{prooftree}
\AxiomC{$\Gamma \vdash e_1 : \tau_1 \rightarrow \tau \, | \, \sigma_1$}
\AxiomC{$\sigma_1(\Gamma) \vdash e_2 : \sigma_1(\tau_1) \, | \, \sigma_2$}
\RightLabel{Application}
\BinaryInfC{$\Gamma \vdash e_1 e_2 : \tau \, | \, \sigma_2 \circ \sigma_1$}
\end{prooftree}

\begin{lstlisting}

    syntax KItem ::= typeInferenceFunAppli(ElemList, Map, K, K, Map) [strict(1)]

    //application rule
    rule <k> typeInferenceFun(.ElemList, Gamma:Map, 'funApp(Eone:K,, Etwo:K), Guess:KItem)
          => typeInferenceFunAppli(val(typeInferenceFun(.ElemList, Gamma, Eone, funtype(guessType(TypeIt),Guess))), .ElemList, Gamma, Etwo, guessType(TypeIt), .Map) ...</k>
         <typeIterator> TypeIt:Int => TypeIt +Int 1 </typeIterator>

    rule <k> typeInferenceFunAppli(valValue(mapBagResult(Sigmaone:Map)), .ElemList, Gamma:Map, Etwo:KItem, guessType(TypeIt:Int), .Map)
          => typeInferenceFunAppli(val(typeInferenceFun(.ElemList, gammaSub(Sigmaone, Gamma, .Map), Etwo, typeSub(Sigmaone, guessType(TypeIt)))), .ElemList, .Map, .K, .K, Sigmaone) ...</k>

    rule <k> typeInferenceFunAppli(valValue(mapBagResult(Sigmatwo:Map)), .ElemList, .Map, .K, .K, Sigmaone:Map)
          => mapBagResult(compose(Sigmatwo, Sigmaone)) ...</k>

\end{lstlisting}

\subsection{IfThenElse Rule}

\begin{prooftree}
\AxiomC{$\Gamma \vdash e_1 : \text{bool} \, | \, \sigma_1$}
\AxiomC{$\sigma_1(\Gamma) \vdash e_2 : \sigma_1(\tau) \, | \, \sigma_2$}
\AxiomC{$\sigma_2 \circ \sigma_1(\Gamma) \vdash e_3 : \sigma_2 \circ \sigma_1(\tau) \, | \, \sigma_3$}
\RightLabel{IfThenElse}
\TrinaryInfC{$\Gamma \vdash \text{if} \, e_1 \, \text{then} \, e_2 \, \text{else} \, e_3 : \tau \, | \, \sigma_3 \circ \sigma_2 \circ \sigma_1$}
\end{prooftree}

\begin{lstlisting}
    syntax KItem ::= typeInferenceFunIfThen(ElemList, Map, K, K, K, Map, Map) [strict(1)]

    //if_then_else rule
    rule <k> typeInferenceFun(.ElemList, Gamma:Map, 'ifThenElse(Eone:K,, Optsem:K,, Etwo:K,, Optsemtwo:K,, Ethree:K), Guess:KItem)
          => typeInferenceFunIfThen(val(typeInferenceFun(.ElemList, Gamma, Eone, Bool)), .ElemList, Gamma, Etwo, Ethree, Guess, .Map, .Map) ...</k>

    rule <k> typeInferenceFunIfThen(valValue(mapBagResult(Sigmaone:Map)), .ElemList, Gamma:Map, Etwo:KItem, Ethree:KItem, Guess:KItem, .Map, .Map)
          => typeInferenceFunIfThen(val(typeInferenceFun(.ElemList, gammaSub(Sigmaone, Gamma, .Map), Etwo, typeSub(Sigmaone, Guess))), .ElemList, Gamma, .K, Ethree, Guess, Sigmaone, .Map) ...</k>

    rule <k> typeInferenceFunIfThen(valValue(mapBagResult(Sigmatwo:Map)), .ElemList, Gamma:Map, .K, Ethree:KItem, Guess:KItem, Sigmaone:Map, .Map)
          => typeInferenceFunIfThen(val(typeInferenceFun(.ElemList, gammaSub(compose(Sigmatwo, Sigmaone), Gamma, .Map), Ethree, typeSub(compose(Sigmatwo, Sigmaone), Guess))), .ElemList, .Map, .K, .K, .K, Sigmaone, Sigmatwo) ...</k>

    rule <k> typeInferenceFunIfThen(valValue(mapBagResult(Sigmathree:Map)), .ElemList, .Map, .K, .K, .K, Sigmaone:Map, Sigmatwo:Map)
          => mapBagResult(compose(compose(Sigmathree, Sigmatwo), Sigmaone)) ...</k>
          
\end{lstlisting}

\subsection{LetIn Rule}

The LetIn Rule must allow for mutual recursion. This means that an example program that must be inferred is

\begin{lstlisting}
let {f = \x -> y x; y = \x -> f x} in (f 2)
\end{lstlisting}

In this example, the f is an expression that refers to y and y is an expression that refers to f.

\begin{prooftree}
\AxiomC{}
\RightLabel{LetIn}
\UnaryInfC{$\Gamma \vdash c : \tau \, | \, \text{unify}\{ ( \tau , \text{freshInstance}(\tau))\}$}
\end{prooftree}

\begin{lstlisting}

    syntax KItem ::= typeInferenceFunLetIn(ElemList, Map, Map, K, K, K, Int, Int, Map, Map) [strict(1)]
    syntax KItem ::= grabLetDeclName(K, Int) [function]
    syntax KItem ::= grabLetDeclExp(K, Int) [function]
    syntax KItem ::= mapLookup(Map, K) [function]
    syntax Map ::= makeDeclMap(K, Int, Map) [function]
    syntax Map ::= applyGEN(Map, Map, Map, Map) [function]

    //Haskell let in rule (let rec in exp + let in rule combined)
    //gamma |- let rec f1 = e1 and f2 = e2 and f3 = e3 .... in e =>
    //beta, [f1 -> tau1, f2 -> tau2, f3 -> tau3,....] + gamma |- e1 : tau1 | sigma1,  [f1 -> simga1(tau1), f2 -> sigma1(tau2), f3 -> sigma1(tau3),....] + sigma1(gamma) |- e2 : sigma1(tau2) | sigma2  [f1 -> sigma2 o sigma1(tau1), f2 -> sigma2 o sigma1(tau2), f3 -> sigma2 o sigma1(tau3),....] + sigma2 o sigma1(gamma) |- e3 : sigma2 o sigma1(tau3) .....  [f1 -> gen(sigma_n o sigma2 o sigma1(tau1), sigma_n o sigma2 o sigma1(Gamma)), f2 -> gen(tau2), f3 -> gen(tau3),....] + gamma |- e : something
    rule <k> typeInferenceFun(.ElemList, Gamma:Map, 'letIn(D:K,, E:K), Guess:KItem)
          => typeInferenceFunLetIn(.ElemList, Gamma, makeDeclMap(D, TypeIt, .Map), D, E, Guess, 0, TypeIt, .Map, Beta) ...</k>
         <typeIterator> TypeIt:Int => TypeIt +Int size(makeDeclMap(D, TypeIt, .Map)) </typeIterator>
         <tempBeta> Beta:Map </tempBeta>

    rule <k> typeInferenceFunLetIn(.ElemList, Gamma:Map, DeclMap:Map, D:KItem, E:KItem, Guess:KItem, Iter:Int, TypeIt:Int, OldSigma:Map, Beta:Map)
           => typeInferenceFunLetIn(val(typeInferenceFun(.ElemList, Gamma DeclMap, grabLetDeclExp(D, Iter), mapLookup(DeclMap, grabLetDeclName(D, Iter)))), .ElemList, Gamma, DeclMap, D, E, Guess, Iter, TypeIt, OldSigma, Beta) ...</k>
          //=> typeInferenceFunLetIn(val(typeInferenceFun(DeclMap, grabLetDeclExp(D, Iter +Int TypeIt), Guess)), .ElemList, Gamma, DeclMap, D, E, Guess, Iter, TypeIt, OldSigma) ...</k>
          requires Iter <Int (size(DeclMap))

    rule <k> typeInferenceFunLetIn(valValue(mapBagResult(Sigma:Map)), .ElemList, Gamma:Map, DeclMap:Map, D:KItem, E:KItem, Guess:KItem, Iter:Int, TypeIt:Int, OldSigma:Map, Beta:Map)
          => typeInferenceFunLetIn(.ElemList, gammaSub(Sigma,Gamma,.Map), gammaSub(Sigma, DeclMap,.Map), D, E, typeSub(Sigma, Guess), Iter +Int 1, TypeIt, compose(Sigma,OldSigma), Beta) ...</k>
          requires Iter <Int (size(DeclMap))

    rule <k> typeInferenceFunLetIn(.ElemList, Gamma:Map, DeclMap:Map, D:KItem, E:KItem, Guess:KItem, Iter:Int, TypeIt:Int, OldSigma:Map, Beta:Map)
          => typeInferenceFunLetIn(val(typeInferenceFun(.ElemList, Gamma applyGEN(Gamma, DeclMap, .Map, Beta), E, Guess)), .ElemList, Gamma, DeclMap, D, E, Guess, Iter, TypeIt, OldSigma, Beta) ...</k>
          requires Iter >=Int (size(DeclMap))

    rule <k> typeInferenceFunLetIn(valValue(mapBagResult(Sigma:Map)), .ElemList, Gamma:Map, DeclMap:Map, D:KItem, E:KItem, Guess:KItem, Iter:Int, TypeIt:Int, OldSigma:Map, Beta:Map)
          => mapBagResult(compose(Sigma, OldSigma))...</k>
          requires Iter >=Int (size(DeclMap))
          
          
\end{lstlisting}

\begin{lstlisting}

    rule mapLookup((Name |-> Type:KItem) DeclMap:Map, Name:KItem) => Type
    rule mapLookup(DeclMap:Map, Name:KItem) => Name
         requires notBool(Name in keys(DeclMap))

    rule makeDeclMap('decls(Dec:K), TypeIt:Int, NewMap:Map) => makeDeclMap(Dec, TypeIt, NewMap)
    rule makeDeclMap('declsList('declPatRhs('apatVar(Var:K),, Righthand:K),, Rest:K), TypeIt:Int, NewMap:Map) => makeDeclMap('decls(Rest), TypeIt +Int 1, NewMap[Var <- guessType(TypeIt)])
    rule makeDeclMap(.DeclsList, TypeIt:Int, NewMap:Map) => NewMap

    rule grabLetDeclName('decls(Dec:K), Iter:Int) => grabLetDeclName(Dec, Iter)
    rule grabLetDeclName('declsList(Dec:K,, Rest:K), Iter:Int) => grabLetDeclName(Rest, Iter -Int 1)
         requires Iter >Int 0
    rule grabLetDeclName('declsList('declPatRhs('apatVar(Var:K),, Righthand:K),, Rest:K), Iter:Int) => Var
         requires Iter <=Int 0


    rule grabLetDeclExp('decls(Dec:K), Iter:Int) => grabLetDeclExp(Dec, Iter)
    rule grabLetDeclExp('declsList(Dec:K,, Rest:K), Iter:Int) => grabLetDeclExp(Rest, Iter -Int 1)
         requires Iter >Int 0
    rule grabLetDeclExp('declsList('declPatRhs('apatVar(Var:K),, Righthand:K),, Rest:K), Iter:Int) => grabLetDeclExp(Righthand, Iter)
         requires Iter <=Int 0
    rule grabLetDeclExp('eqExpOptDecls(Righthand:K,, Opt:K), Iter:Int) => 'eqExpOptDecls(Righthand,, Opt)

    rule genGamma('apatVar(Vari:K), Gamma:Map, Guess:K) => Gamma[Vari <- Guess]
    rule genGamma('apatCon(Vari:K,, Pattwo:K), Gamma:Map, Guess:K) => Gamma[Vari <- Guess]

    rule genLambda('apatVar(Vari:K), Ex:K) => Ex
    rule genLambda('apatCon(Vari:K,, Pattwo:K), Ex:K) => 'lambdaFun(Pattwo,, Ex)


    rule gammaSub(Sigma:Map, (Key:KItem |-> Type:KItem) Gamma:Map, Newgamma:Map)
        => gammaSub(Sigma, Gamma, Newgamma[Key <- typeSub(Sigma, Type) ] )

    rule gammaSub(Sigma:Map, .Map, Newgamma:Map)
      => Newgamma

    rule freshInstance(guessType(TypeIt:Int), Iter:Int) => guessType(TypeIt)
    rule freshInstance(forAll(.Set, B:K), Iter:Int) => B
    rule freshInstance(forAll(SetItem(C:KItem) A:Set, B:K), Iter:Int) => freshInstance(forAll(A, freshInstanceInner(C, B, Iter)), Iter +Int 1)

    syntax KItem ::= freshInstanceInner(K,K,Int) [function]

    rule freshInstanceInner(Repl:KItem, funtype(A:K, B:K), Iter:Int) => funtype(freshInstanceInner(Repl,A,Iter),freshInstanceInner(Repl,B,Iter))
    rule freshInstanceInner(Repl:KItem, Repl, Iter:Int) => guessType(Iter)
    rule freshInstanceInner(Repl:KItem, Target:KItem, Iter:Int) => Target [owise]

    rule paramSize(forAll(A:Set, B:K)) => size(A)
    rule paramSize(A:K) => 0 [owise]


     rule applyGEN(Gamma:Map, (Key:KItem |-> Type:KItem) DeclMap:Map, NewMap:Map, Beta:Map)
       => applyGEN(Gamma, DeclMap, NewMap[Key <- gen(Gamma, Type, Beta)], Beta)

     rule applyGEN(Gamma:Map, .Map, NewMap:Map, Beta:Map)
       => NewMap

    //GEN
    //GEN(Gamma, Tau) => Forall alpha

    syntax KItem ::= gen(Map, K, Map) [function]
    syntax Set ::= freeVarsTy(K, Map) [function]
    syntax Set ::= freeVarsEnv(Map, Map) [function]


    rule gen(Gamma:Map, forAll(Para:Set, Tau:KItem), Beta:Map) => forAll(freeVarsTy(forAll(Para:Set, Tau), Beta) -Set freeVarsEnv(Gamma, Beta), Tau)
    rule gen(Gamma:Map, Tau:KItem, Beta:Map) => forAll(freeVarsTy(Tau, Beta) -Set freeVarsEnv(Gamma, Beta), Tau) [owise]

    //rule gen(Gamma:Map, forAll(Para:Set, Tau:KItem), Beta:Map) => forAll(freeVarsTy(forAll(Para:Set, Tau), Beta) -Set freeVarsEnv(Gamma, Beta), Tau)

    rule freeVarsTy(guessType(TypeIt:Int), Beta:Map) => SetItem(guessType(TypeIt:Int))
    rule freeVarsTy(funtype(Tauone:KItem, Tautwo:KItem), Beta:Map) => freeVarsTy(Tauone, Beta) freeVarsTy(Tautwo, Beta)
    rule freeVarsTy(Tau:KItem, Beta:Map) => .Set
         requires (forAll(.Set, Tau)) in values(Beta)
    rule freeVarsTy(forAll(Para:Set, Tau:KItem), Beta:Map) => freeVarsTy(Tau, Beta) -Set Para
    rule freeVarsEnv(Gamma:Map, Beta:Map) => listToSet(values(Beta), .Set)
\end{lstlisting}
\section{Unification Algorithm}

Let $S = \{(s_1 , t_1), (s_2 , t_2), \cdots, (s_n , t_n)\}$ be a
unification problem.

Case $S = \{ \} : \text{Unif}(S) = \text{Identity function}$ (ie
no substitution)

Case $S = {(s, t)} \cup S')$: Four main steps

Delete: if $s = t$ (they are the same term)
then $Unif(S) = Unif(S')$

Decompose: if $s = f(q_1, \cdots, q_m )$ and
$t = f(r_1, \cdots, r_m )$ (same f, same m!), then
$\text{Unif}(S) = \text{Unif}(\{(q_1 , r_1 ), ..., (q_m , r_m )\} \cup S')$

Orient: if $t = x$ is a variable, and s is not a
variable, $\text{Unif}(S) = \text{Unif} (\{(x,s)\} \cup S')$

Eliminate: if s = x is a variable, and
x does not occur in t (the occurs
check), then

Let $\phi = x \mapsto t$

Let $\psi = \text{Unif}(\phi(S'))$

$\text{Unif}(S) = \{x \mapsto \psi(t) \} \circ \psi$

Note: {x |→ a} o {y |→ b} =
{y |→ ({x |→ a}(b)} o {x |→ a} if y not
in a

[cs 421 class notes]
\begin{lstlisting}
    //Unification

    syntax Map ::= uniFun(List) [function]
    syntax Bool ::= isVarType(K) [function]
    syntax Bool ::= notChildVar(K,K) [function]
    syntax KItem ::= uniPair(K,K)

    syntax List ::= uniSub(Map,K) [function] //apply substitution to unification

    syntax KItem ::= typeSub(Map,K) [function] //apply substitution to type
    syntax Map ::= compose(Map,Map) [function]

    rule uniFun(.List) => .Map //substi(.K,.K) is id substitution

    rule uniFun(ListItem(uniPair(S:K,S)) Rest:List) => uniFun(Rest)  //delete rule

    rule uniFun(ListItem(uniPair(S:K,T:K)) Rest:List) => uniFun(ListItem(uniPair(T,S)) Rest) //orient rule
         requires isVarType(T) andBool (notBool isVarType(S))

    rule uniFun(ListItem(uniPair(funtype(A:K, B:K), funtype(C:K, D:K))) Rest:List) => uniFun(ListItem(uniPair(A, C)) ListItem(uniPair(B, D)) Rest:List) //decompose rule function type

    rule uniFun(ListItem(uniPair(S:K,T:K)) Rest:List) 
      => compose((S |-> typeSub(uniFun(uniSub((S |-> T),Rest)),T)),uniFun(uniSub((S |-> T),Rest))) //eliminate rule
    //  => compose(uniFun(uniSub((S |-> T),Rest)),(S |-> typeSub(uniFun(uniSub((S |-> T),Rest)),T))) //eliminate rule
         requires isVarType(S) andBool notChildVar(S,T)

    rule isVarType(S:K) => true
         requires getKLabel(S) ==KLabel 'guessType
    rule isVarType(S:K) => false [owise]

    rule notChildVar(S:K,T:K) => true

    rule uniSub(Sigma:Map,.List) => .List
    rule uniSub(.Map,L:List) => L
    rule uniSub(Sigma:Map, Rest:List ListItem(uniPair(A:K, B:K))) => uniSub(Sigma, Rest) ListItem(uniPair(typeSub(Sigma, A), typeSub(Sigma, B)))

    //rule typeSub(substi(.Map),Tau:KItem) => Tau
    rule typeSub(Sigma:Map (Tau |-> Newtau:KItem),Tau:KItem) => typeSub(Sigma (Tau |-> Newtau),Newtau)
    rule typeSub(Sigma:Map,funtype(Tauone:KItem,Tautwo:KItem)) => funtype(typeSub(Sigma,Tauone),typeSub(Sigma,Tautwo))
    rule typeSub(Sigma:Map,Tau:KItem) => Tau [owise]

    syntax Map ::= composeIn(Map, Map, Map, K, K) [function]

    rule compose(Sigmaone:Map, Sigmatwo:Map) => composeIn(Sigmaone, Sigmatwo, .Map, .K, .K)

    rule composeIn(Sigmaone:Map, (Key:KItem |-> Type:KItem) Sigmatwo:Map, NewMap:Map, .K, .K) => composeIn(Sigmaone, Sigmatwo, NewMap, Key, Type)

    rule composeIn((Keyone |-> Typetwo:KItem) Sigmaone:Map, Sigmatwo:Map, NewMap:Map, Keyone:KItem, Typeone:KItem) => composeIn(Sigmaone, Sigmatwo, NewMap, Keyone, Typeone)

    rule composeIn((Typeone |-> Typetwo:KItem) Sigmaone:Map, Sigmatwo:Map, NewMap:Map, Keyone:KItem, Typeone:KItem) => composeIn((Typeone |-> Typetwo) Sigmaone, Sigmatwo, NewMap[Keyone <- Typetwo], .K, .K)
         requires notBool(Keyone in keys(Sigmaone))

    rule composeIn(Sigmaone:Map, Sigmatwo:Map, NewMap:Map, Keyone:KItem, Typeone:KItem) => composeIn(Sigmaone, Sigmatwo, NewMap[Keyone <- Typeone], .K, .K) [owise]

    rule composeIn(Sigmaone:Map, .Map, NewMap:Map, .K, .K) => Sigmaone NewMap
endmodule
\end{lstlisting}