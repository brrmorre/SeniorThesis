\chapter{Context Sensitive Checks}
I also placed the user defined types into data structures in order to perform several checks to make sure that the user did not have errors when creating types. Then the data structures will be transformed into a form that will be used for type inferencing.
Section 4 of the Haskell 2010 report specifies the haskell type system.
%https://www.haskell.org/onlinereport/haskell2010/haskellch4.html#x10-620004
In the topdecl sort, there are three typecons that are used to create user defined datatypes. Data, type, and newtype. The end goal is to put the user defined types into a data structure which I can use to perform type inferencing. These three typecons are used to create user defined types.
The first one is type,
type simpletype = type 
This is used in a haskell program to declare a new type as a single type. In effect, it renames the type where both names now can be used to refer to the type.
type Username = String
Is one such example usage of type, it creates a new type Username, which is defined as just a string. Now the programmer can refer to Username or String to make a string.
	The second one is data,
	\begin{lstlisting}
data [context =>] simpletype [= constrs] [deriving]
	\end{lstlisting}
This allows a user to declare a new type that may include many fields, and polymorphic types. For instance:
data Date = Date Int Int Int
This is a new type that includes the typecon ‘Date’ followed by three integers.
data Poly a = Number a
This is a new polymorphic type with polymorphic parameter a, that has the typecon ‘Number’.
	The third one is newtype,
	\begin{lstlisting}
newtype [context =>] simpletype = newconstr [deriving]
\end{lstlisting}
This is very similar to data except it only parses when the newtype has only one typecon and one field.

I perform several checks here, 
1. The programmer should not be able to make two user defined datatypes with the same name, even if one is created using ‘data’ and another is created using ‘type’ for instance.
2. The programmer should not be able to use the same typecons when making different options for their types or use the same typecons for different types.
3. There should be no cycles in type renaming using ‘type’, and the type renaming chains using ‘type’ should terminate with a type defined with ‘data’ or ‘newtype’.
4. The argument sorts for types defined using ‘data’ or ‘newtype’ should be types that exist.
5. The polymorphic parameters that appear on the right hand side of a ‘data’ declaration need to appear on the left hand side as well.
6. The polymorphic parameters that appear on the left hand side of a ‘data’ declaration need to be unique.
I implemented a map, called alpha, of new type names as the keys and their declared types as the entries. I then collected all appearances of the typecon ‘type’ in the program, and put simpletype -> type in the alpha map. However, one of the things I needed to check for in the program was whether a user declared multiple definitions with ‘type’, so I could not use a map in K because they only allow unique keys with unique entries. So I initially used a set of tuples, and then changed it to a map after checking for multiple type declarations.


The second data structure I made is called T. T holds the user defined types created using ‘data’ and ‘newtype’.
\begin{lstlisting}
syntax KItem ::= TList(K) 
//list of T objects for every new type introduced by data and newtype
syntax KItem ::= TObject(K,K,K)
//(type name, entire list of poly type vars, list of inner T pieces)
syntax KItem ::= InnerTPiece(K,K,K,K,K)
//(type constructor, poly type vars, argument sorts, entire constr block, type name)
\end{lstlisting}
T is a list of TObjects, each TObject represents a single user defined datatype. It holds the name, the list of polymorphic parameters, and a list of inner T pieces.
An inner T piece represents an option of what a type could be. It consists of a type constructor, a list of polymorphic parameters required for this option, the fields for this option, the entire subtree of the AST for this option unedited, and the type name again.
I then used these data structures to perform these checks, and afterwards will transform them into a new data structure to perform type inferencing.

\begin{lstlisting}
//
requires "haskell-syntax.k"
requires "haskell-configuration.k"

module HASKELL-PREPROCESSING
    imports HASKELL-SYNTAX
    imports HASKELL-CONFIGURATION

    //USER DEFINED LIST
    //definition of ElemList

    //syntax KItem ::= ElemList
    syntax ElemList ::= List{Element,","} [strict]
//    syntax Int ::= lengthOfList(ElemList) [function]

//    rule lengthOfList(.ElemList) => 0
//    rule lengthOfList(val(K:K),L:ElemList) => 1 +Int lengthOfList(L)
//    rule lengthOfList(valValue(K:K),L:ElemList) => 1 +Int lengthOfList(L)

    syntax Element ::= val(K) [strict]
    syntax ElementResult ::= valValue(K)
    syntax Element ::= ElementResult
    syntax KResult ::= ElementResult
    rule val(K:KResult) => valValue(K) [structural]

    //form ElemList
//    syntax ElemList ::= formElemList(K) [function]

    //CONVERT ~> TO List
    //list convert
//    syntax List ::= convertToList(K)  [function]
//    rule convertToList(.K) => .List
//    rule convertToList(A:KItem ~> B:K) => ListItem(A) convertToList(B)


    syntax KItem ::= dealWithImports(K,K)

    rule <k> 'modListSingle('module(A:K,, B:K)) => dealWithImports(A,B) ...</k>

    (.Bag =>
          <module>...   //DOT DOT DOT MEANS OVERWRITE ONLY SOME OF THE DEFAULTS
    	    <moduleName> A </moduleName>
       ...</module>
    )

    rule <k> 'modList('module(A:K,, B:K),, C:K) => dealWithImports(A,B) ~> C ...</k>

    (.Bag =>
          <module>...   //DOT DOT DOT MEANS OVERWRITE ONLY SOME OF THE DEFAULTS
    	    <moduleName> A </moduleName>
       ...</module>
    )

//    rule dealWithImports(Mod:K, A:K) => callInit(A)

//    rule <k> dealWithImports(Mod:K, A:K) => callInit(A) ...</k>

    rule <k> dealWithImports(Mod:K, 'bodyimpandtop(A:K,, B:K)) => .K ...</k>
        <importTree> L:List => L importListConvert(Mod, A) </importTree>
        <recurImportTree> L:List => L importListConvert(Mod, A) </recurImportTree>

        <moduleName> Mod </moduleName>
        <imports> S:Set (.Set => SetItem(A)) </imports>
        <moduleTempCode> OldTemp:K => B </moduleTempCode>

    rule <k> dealWithImports(Mod:K, 'bodyimpdecls(A:K)) => .K ...</k>
        <importTree> L:List => L importListConvert(Mod, A) </importTree>
        <recurImportTree> L:List => L importListConvert(Mod, A) </recurImportTree>

        <moduleName> Mod </moduleName>
        <imports> S:Set (.Set => SetItem(A)) </imports>

//    rule <k> dealWithImports(Mod:K, 'bodytopdecls(A:K)) => callInit(A) ...</k>
    rule <k> dealWithImports(Mod:K, 'bodytopdecls(B:K)) => .K ...</k>

        <moduleName> Mod </moduleName>
        <moduleTempCode> OldTemp:K => B </moduleTempCode>

    //importlist convert
    syntax List ::= importListConvert(K,K) [function]
    syntax KItem ::= impObject(K,K)

    rule importListConvert(Name:K, 'impDecls(A:K,, Rest:K)) => importListConvert(Name, A) importListConvert(Name, Rest)
    rule importListConvert('moduleName(Name:K), 'impDecl(A:K,, Modid:K,, C:K,, D:K)) => ListItem(impObject(Name, Modid))
    rule importListConvert(Name:K, .ImpDecls) => .List

////////////////////////////////////////////////////////////////////////////////////////////////////////////////////////

    /*NEW TODO ALGORITHM
1. Construct tree for module inclusion
2. Check tree for cycles
3. Go to each leaf and recursively go up the tree and build alpha* and beta* for the types of the module and the children
(and specify scoping) (desugar the scope so that each type specifies the scope) */

    syntax KItem ::= "checkImportCycle"
    syntax KItem ::= "recurseImportTree"

/*    rule <k> performNextChecks
             => checkUseVars
                ~> (checkLabelUses
                ~> (checkBlockAddress(.K)
                ~> (checkNoNormalBlocksHavingLandingpad(.K, TNS -Set TES)
                ~> (checkAllExpBlocksHavingLandingpad(.K, TES)
                ~> (checkAllExpInFromInvoke(.K, TES)
                ~> (checkLandingpad
                ~> checkLandingDomResumes)))))) ...</k> */

    rule <k> startImportRecursion => checkImportCycle 
                                     ~> (recurseImportTree)...</k>

    syntax KItem ::= cycleCheck(K,Map,List,List) [function] //current node, map of all nodes to visited or not, stack, graph
    syntax Map ::= createVisitMap(List,Map) [function] //graph, visitmap
    syntax KItem ::= getUnvisitedNode(K,K, Map) [function] //visitmap
    syntax List ::= getNodeNeighbors(K,List) [function] //visitmap

    rule <k> checkImportCycle
             => cycleCheck(.K,createVisitMap(I, .Map),.List,I) ...</k>
         <importTree> I:List </importTree>
         <impTreeVMap> .Map => createVisitMap(I, .Map) </impTreeVMap>

    syntax KItem ::= "visited"
    syntax KItem ::= "unvisited"
    syntax KItem ::= "none"

    rule createVisitMap(ListItem(impObject(A:K,B:K)) Rest:List, M:Map) 
             => createVisitMap(Rest, M[A <- unvisited][B <- unvisited])
    rule createVisitMap(.List, M:Map) => M

    rule getUnvisitedNode(.K, .K, .Map) => none
    rule getUnvisitedNode(.K, .K, (A:K |-> B:K) M:Map)
           => getUnvisitedNode(A, B, M)
    rule getUnvisitedNode(A:KItem, unvisited, M:Map) => A
    rule getUnvisitedNode(A:KItem, visited, M:Map)
            =>  getUnvisitedNode(.K, .K, M)



    rule getNodeNeighbors(Node:K,.List) => .List
    rule getNodeNeighbors(.K,Rest:List) => .List

    rule getNodeNeighbors(Node:KItem,ListItem(impObject(Node,B:KItem)) Rest:List) => getNodeNeighbors(Node, Rest) ListItem(B)
    rule getNodeNeighbors(Node:KItem,ListItem(impObject(A:KItem,B:KItem)) Rest:List) => getNodeNeighbors(Node, Rest)
         requires Node =/=K A


    rule cycleCheck(none, M:Map, .List, L:List) => .K
    rule cycleCheck(.K, M:Map, .List, I:List) => cycleCheck(getUnvisitedNode(.K, .K, M), M, .List, I)
    rule cycleCheck(.K, M:Map, ListItem(Node:K) S:List, I:List) => cycleCheck(Node, M, S, I)
    rule cycleCheck(Node:K, M:Map, S:List, I:List)
                 => cycleCheck(.K, M[Node <- visited], getNodeNeighbors(Node,I) S, I)
         requires Node =/=K .K andBool Node =/=K none
    rule cycleCheck(.K, M:Map, ListItem(Node:K) S:List, I:K) => cycleCheck(Node, M, S, I)
         requires S =/=K .List

/*
    rule cycleCheck(A:K,.K,.K,I:K) => cycleCheck(A,createVisitMap(I,.Map),.List,I)



    rule cycleCheck(Node:K, M:Map, S:List, I:K) => cycleCheck(.K, M[Node <- visited], getNodeNeighbors(Node,I) S, I)

    rule cycleCheck(.K, M:Map, ListItem(Node:K) S:List, I:K) => cycleCheck(Node, M, S, I)
    //rule cycleCheck(.K, M:Map, .K, ListItem(impObject(A:K,B:K)) Rest:List) => cycleCheck(ListItem(impObject(A:K,B:K)) Rest:List)
*/

////////////////////////////////////////////////////////////////////////////////////////////////////////////////////////

//COPY IMPORT GRAPH, NEED SECOND GRAPH FOR RECURSING, ADDITIONAL GRAPH FOR SELECTING IMPORTS FOR ALPHA* AND BETA*
//DFS for leaf
//acquire alpha and beta for leaf
//merge alpha and beta with imports to produce alpha* and beta*
//perform checks
//perform inferencing
//insert alpha* and beta* into importing modules
//remove all edges pointing to leaf

    syntax KItem ::= "leafDFS"
    syntax KItem ::= "getAlphaAndBeta"
    syntax KItem ::= "getAlphaBetaStar"
    syntax KItem ::= "performIndividualChecks"
    syntax KItem ::= "performIndividualInferencing"
    syntax KItem ::= "insertAlphaBetaStar"
    syntax KItem ::= "removeAllEdges"
    syntax KItem ::= "seeIfFinished"

    rule <k> recurseImportTree => leafDFS 
                                  ~> (getAlphaAndBeta
                                  //~> (getAlphaBetaStar
                                  ~> (performIndividualInferencing))...</k>

//rule <k> dealWithImports(Mod:K, 'bodytopdecls(A:K)) => callInit(A) ...</k>

//    rule <k> leaf
//             => cycleCheck(.K,createVisitMap(I, .Map),.List,I) ...</k>
//         <importTree> I:List </importTree>

////////////////////////////////////////////////////////////////////////////////////////////////////////////////////////

    syntax KItem ::= returnLeafDFS(K,List,Map) [function] //current node, map of all nodes to visited or not, stack, graph
    syntax KItem ::= innerLeafDFS(K,List) [function]
    syntax KItem ::= loadModule(K)

    rule <k> leafDFS
             => returnLeafDFS(.K,I,M) ...</k>
         <recurImportTree> I:List </recurImportTree>
         <impTreeVMap> M:Map </impTreeVMap>

    rule returnLeafDFS(.K,ListItem(impObject(Node:KItem,B:KItem)) I:List,M:Map) => returnLeafDFS(B,I,M)
    rule returnLeafDFS(Node:KItem,I:List,M:Map) => returnLeafDFS(innerLeafDFS(Node,I),I,M)
         requires innerLeafDFS(Node,I) =/=K none
    rule returnLeafDFS(Node:KItem,I:List,M:Map) => loadModule(Node)
         requires innerLeafDFS(Node,I) ==K none

    rule innerLeafDFS(Node:KItem,ListItem(impObject(Node,B:KItem)) I:List) => B
    rule innerLeafDFS(Node:KItem,ListItem(impObject(A:KItem,B:KItem)) I:List) => innerLeafDFS(Node,I)
         requires Node =/=K A
    rule innerLeafDFS(Node:KItem,.List) => none
//    returnLeafDFS(Node:KItem,ListItem(impObject(Node,B:KItem)) I:List,M:Map) => returnLeafDFS(B,I,M)


////////////////////////////////////////////////////////////////////////////////////////////////////////////////////////

    //call before Checker Code
//    rule <k> callInit(S:K) => initPreModule(S) ...</k>
//         <tempModule> A:K => S </tempModule>

    rule <k> loadModule(S:KItem) => .K ...</k>
         <tempModule> A:K => S </tempModule>

    rule <k> getAlphaAndBeta => initPreModule(Code) ...</k>
         <tempModule> Mod:KItem </tempModule>

         <moduleName> 'moduleName(Mod) </moduleName>
         <moduleTempCode> Code:KItem </moduleTempCode>


////////////////////////////////////////////////////////////////////////////////////////////////////////////////////////
    //get alpha and beta
    syntax KItem ::= Module(K, K)
    syntax KItem ::= preModule(K,K) //(alpha, T)

    // STEP 1 CONSTRUCT T AND ALPHA
    // alpha = type
    // T = newtype and data, temporary data structure

    syntax KItem ::= initPreModule(K) [function]
    syntax KItem ::= getPreModule(K, K) [function] //(Current term, premodule)
    syntax KItem ::= makeT (K,K,K,K)

    syntax KItem ::= fetchTypes (K,K,K,K)

    syntax List ::= makeInnerT (K,K,K) [function] //LIST
    syntax List ::= getTypeVars(K) [function] //LIST

    syntax KItem ::= getCon(K) [function]
    syntax List ::= getArgSorts(K) [function] //LIST

    syntax KItem ::= AList(K)
    syntax KItem ::= AObject(K,K) //(1st -> 2nd) map without idempotency
    syntax KItem ::= ModPlusType(K,K)

    syntax KItem ::= TList(K) //list of T objects for every new type introduced by data and newtype
    syntax KItem ::= TObject(K,K,K,K) //(module name, type name, entire list of poly type vars, list of inner T pieces)
    syntax KItem ::= InnerTPiece(K,K,K,K,K) //(type constructor, poly type vars, argument sorts, entire constr block, type name)

//    rule initPreModule('module(I:ModuleName,, J:K)) => getPreModule(J,preModule(AList(.List),TList(.List)))
//    rule initPreModule('moduleExp(I:ModuleName,, L:K,, J:K)) => getPreModule(J,preModule(AList(.List),TList(.List)))
//    rule initPreModule('moduleBody(J:Body)) => getPreModule(J,preModule(AList(.List),TList(.List)))

    rule initPreModule(J:K) => getPreModule(J,preModule(AList(.List),TList(.List)))

    rule getPreModule('bodytopdecls(I:K), J:K) => getPreModule(I,J)
    rule getPreModule('topdeclslist('type(A:K,, B:K),, Rest:K),J:K) => fetchTypes(A,B,Rest,J) //constructalpha


    rule getPreModule('topdeclslist('data(A:K,, B:K,, C:K,, D:K),, Rest:K),J:K) => makeT(B,C,Rest,J)
    rule getPreModule('topdeclslist('newtype(A:K,, B:K,, C:K,, D:K),, Rest:K),J:K) => makeT(B,C,Rest,J)


    rule getPreModule('topdeclslist('topdecldecl(A:K),, Rest:K),J:K) => getPreModule(Rest,J)
    rule getPreModule('topdeclslist('class(A:K,, B:K,, C:K,, D:K),, Rest:K),J:K) => getPreModule(Rest,J)
    rule getPreModule('topdeclslist('instance(A:K,, B:K,, C:K,, D:K),, Rest:K),J:K) => getPreModule(Rest,J)
    rule getPreModule('topdeclslist('default(A:K,, B:K,, C:K,, D:K),, Rest:K),J:K) => getPreModule(Rest,J)
    rule getPreModule('topdeclslist('foreign(A:K,, B:K,, C:K,, D:K),, Rest:K),J:K) => getPreModule(Rest,J)
    rule getPreModule(.TopDecls,J:K) => J

    //rule getPreModule('module(I:ModuleName,L:K, J:K)) => preModule(J)

    rule <k> fetchTypes('simpleTypeCon(I:TyCon,, H:TyVars), 'atypeGTyCon(C:K), Rest:K, preModule(AList(M:List), L:K)) => getPreModule(Rest,preModule(AList(ListItem(AObject(ModPlusType(ModName,I),C)) M), L)) ...</k>
         <tempModule> ModName:KItem </tempModule>

    rule <k> makeT('simpleTypeCon(I:TyCon,, H:TyVars), D:K, Rest:K, preModule(AList(M:List), TList(ListInside:List))) => getPreModule(Rest,preModule(AList(M),TList(ListItem(TObject(ModName,I,H,makeInnerT(I,H,D))) ListInside))) ...</k>
         <tempModule> ModName:KItem </tempModule>

    rule makeInnerT(A:K,B:K,'nonemptyConstrs(C:K)) => makeInnerT(A,B,C)
    rule makeInnerT(A:K,B:K,'singleConstr(C:K)) => ListItem(InnerTPiece(getCon(C),getTypeVars(C),getArgSorts(C),C,A))
    rule makeInnerT(A:K,B:K,'multConstr(C:K,, D:K)) => ListItem(InnerTPiece(getCon(C),getTypeVars(C),getArgSorts(C),C,A)) makeInnerT(A,B,D)

    rule getTypeVars('constrCon(A:K,, B:K)) => getTypeVars(B)
    rule getTypeVars('optBangATypes(A:K,, Rest:K)) => getTypeVars(A) getTypeVars(Rest)
    rule getTypeVars('optBangAType('emptyBang(.KList),, Rest:K)) => getTypeVars(Rest)
    rule getTypeVars('atypeGTyCon(A:K)) => .List
    rule getTypeVars('atypeTyVar(A:K)) => ListItem(A)
    rule getTypeVars(.OptBangATypes) => .List

    //rule getCon('emptyConstrs()) => .K
    //rule getCon('nonemptyConstrs(A:K)) => getCon(A)
    rule getCon('constrCon(A:K,, B:K)) => A

    //rule getArgSorts('constrCon(A:K,, B:K)) => B
    rule getArgSorts('constrCon(A:K,, B:K)) => getArgSorts(B)
    rule getArgSorts('optBangATypes(A:K,, Rest:K)) => getArgSorts(A) getArgSorts(Rest)
    rule getArgSorts('optBangAType('emptyBang(.KList),, Rest:K)) => getArgSorts(Rest)
    rule getArgSorts('atypeGTyCon(A:K)) => ListItem(A)
    rule getArgSorts('atypeTyVar(A:K)) => .List
    rule getArgSorts(.OptBangATypes) => .List

////////////////////////////////////////////////////////////////////////////////////////////////////////////////////////

    rule <k> preModule(A:K,T:K) => startTTransform ...</k>
         <tempAlpha> OldAlpha:K => A </tempAlpha>
         <tempT> OldT:K => T </tempT>

////////////////////////////////////////////////////////////////////////////////////////////////////////////////////////

    // STEP 2 PERFORM CHECKS

    syntax KItem ::= "error"

    syntax KItem ::= "startChecks"
    syntax KItem ::= "checkNoSameKey"
        //Keys of alpha and keys of T should be unique
    syntax KItem ::= "checkTypeConsDontCollide"
        //Make sure typeconstructors do not collide in T
    syntax KItem ::= "makeAlphaMap"
        //make map for alpha
    syntax KItem ::= "checkAlphaNoLoops"
        //alpha check for no loops
        //check alpha to make sure that everything points to a T
    syntax KItem ::= "checkArgSortsAreTargets"
           //Make sure argument sorts [U] [W,V] are in the set of keys of alpha and targets of T, (keys of T)
    syntax KItem ::= "checkParUsed"
//NEED TO CHECK all the polymorphic parameters from right appear on left. RIGHT SIDE ONLY
//NEED TO CHECK UNIQUENESS FOR POLY PARAM ON LEFT SIDE ONLY

//    rule <k> preModule(A:K,T:K) => startChecks ...</k>
//         <tempAlpha> OldAlpha:K => A </tempAlpha>
//         <tempT> OldT:K => T </tempT>


/*    rule <k> performNextChecks
             => checkUseVars
                ~> (checkLabelUses
                ~> (checkBlockAddress(.K)
                ~> (checkNoNormalBlocksHavingLandingpad(.K, TNS -Set TES)
                ~> (checkAllExpBlocksHavingLandingpad(.K, TES)
                ~> (checkAllExpInFromInvoke(.K, TES)
                ~> (checkLandingpad
                ~> checkLandingDomResumes)))))) ...</k> */

    rule <k> startChecks
             => checkNoSameKey
                ~> (checkTypeConsDontCollide
                ~> (makeAlphaMap
                ~> (checkAlphaNoLoops
                ~> (checkArgSortsAreTargets
                ~> (checkParUsed))))) ...</k>

    rule <k> checkTypeConsDontCollide
             => tyConCollCheck(T,.List,.Set) ...</k>
         <tempT> T:K </tempT>

    //syntax KItem ::= tChecker(K) [function]
    syntax KItem ::= tyConCollCheck(K,K,K) [function] //(TList,List of Tycons,Set of Tycons) 
    syntax KItem ::= lengthCheck(K,K) [function]
    //syntax KItem ::= tyConCollCheck(K,K,K) [function]
    //syntax K ::= innerCollCheck(K) [function]
    //syntax K ::= tyConCollCheckPasser(K, K) [function]

    //rule tChecker(preModule(Alpha:Map,T:K,Mod:K)) => tyConCollCheck(innerCollCheck(T),preModule(Alpha,T,Mod))

    //rule tyConCollCheck(.K,preModule(Alpha:Map,H:K,Mod:K)) => tyConCollCheck(innerCollCheck(H),preModule(Alpha,H,Mod))

    rule tyConCollCheck(TList(ListItem(TObject(ModName:K, A:K,B:K,ListItem(InnerTPiece(Ty:K,E:K,F:K,H:K,G:K)) Inners:List)) Rest:List),J:List,D:Set) => 
                    tyConCollCheck(TList(ListItem(TObject(ModName,A,B,Inners)) Rest),ListItem(Ty) J, SetItem(Ty) D)
    rule tyConCollCheck(TList(ListItem(TObject(ModName:K, A:K,B:K,.List)) Rest:List),J:List,D:Set) => 
                    tyConCollCheck(TList(Rest),J,D)
    rule tyConCollCheck(TList(.List),J:List,D:Set) => 
                    lengthCheck(size(J),size(D))

    rule lengthCheck(A:Int, B:Int) => .K
                    requires A ==Int B

    rule lengthCheck(A:Int, B:Int) => error
                    requires A =/=Int B

    //rule tyConCollCheck(TList(TObject(A:K,B:K,C:K) ~> Rest:K),J:K) => tyConCollCheckPasser(TList(innerCollCheck(TObject(A:K,B:K,C:K)) ~> Rest:K),J:K)
    syntax KItem ::= keyCheck(K,K,K,K) [function] //(Alpha, T, List of names, Set of names)

    rule <k> checkNoSameKey
             => keyCheck(A, T, .Set, .List) ...</k>
         <tempAlpha> A:K </tempAlpha>
         <tempT> T:K </tempT>
    //rule <k> checkAlphaNoSameKey
    //         => akeyCheck(.K, .Set) ...</k>

    rule keyCheck(AList(ListItem(AObject(A:K,B:K)) C:List), T:K, D:Set, G:List) => keyCheck(AList(C), T, SetItem(A) D, ListItem(A) G)
    rule keyCheck(AList(.List), TList(ListItem(TObject(ModName:K, A:K,B:K,C:K)) Rest:List), D:Set, G:List) => keyCheck(AList(.List), TList(Rest), SetItem(A) D, ListItem(A) G)
    rule keyCheck(AList(.List), TList(.List), D:Set, G:List) => lengthCheck(size(G),size(D))


    syntax KItem ::= makeAlphaM(K,K) [function] //(Alpha, AlphaMap)
    syntax KItem ::= tAlphaMap(K) //(AlphaMap) temp alphamap

    rule <k> makeAlphaMap
             => makeAlphaM(A, .Map) ...</k>
         <tempAlpha> A:K </tempAlpha>

    rule makeAlphaM(AList(ListItem(AObject(A:K,B:K)) C:List), M:Map) => makeAlphaM(AList(C), M[A <- B])
    rule makeAlphaM(AList(.List), M:Map) => tAlphaMap(M)

    rule <k> tAlphaMap(M:K) => .K ...</k>
         <tempAlphaMap> OldAlphaMap:K => M </tempAlphaMap>

//    syntax KItem ::= tkeyCheck(K,K,K) [function] //(T,List of T,Set of T)

//    rule <k> checkTNoSameKey
//             => tkeyCheck(T, .Set, T) ...</k>
//         <tempT> T:K </tempT>

//    rule tkeyCheck(TList(ListItem(TObject(A:K,B:K,C:K)) Rest:List), D:Set, G:K) => tkeyCheck(TList(Rest), SetItem(A) D, G)
//    rule tkeyCheck(TList(.List), D:Set, TList(G:List)) => lengthCheck(size(G),size(D))

    syntax KItem ::= aloopCheck(K,K,K,K,K,K,K) [function] //(Alpha,List of Alpha,Set of Alpha,CurrNode,lengthcheck,T,BigSet)

    rule <k> checkAlphaNoLoops
             => aloopCheck(A,.List,.Set,.K,.K,T,.Set) ...</k>
         <tempAlphaMap> A:K </tempAlphaMap>
         <tempT> T:K </tempT>

    //aloopCheck set and list to check cycles
    rule aloopCheck(Alpha:Map (A:KItem |-> B:KItem), D:List, G:Set, .K, .K,T:K,S:Set) => aloopCheck(Alpha, ListItem(B) ListItem(A) D, SetItem(B) SetItem(A) G, B, .K,T,S)
    rule aloopCheck(Alpha:Map (H |-> B:KItem), D:List, G:Set, H:KItem, .K,T:K,S:Set) => aloopCheck(Alpha, ListItem(B) D, SetItem(B) G, B, .K,T,S)

    rule aloopCheck(Alpha:Map, D:List, G:Set, H:KItem, .K,T:K,S:Set) => aloopCheck(Alpha, .List, .Set, .K, lengthCheck(size(G),size(D)),T,G S) //type rename loop ERROR
         requires (notBool H in keys(Alpha)) andBool (H in typeSet(T, .Set) orBool H in S) 

    rule aloopCheck(Alpha:Map, D:List, G:Set, H:KItem, .K,T:K,S:Set) => error //terminal alpha rename is not in T ERROR
         requires (notBool H in keys(Alpha)) andBool (notBool (H in typeSet(T, .Set) orBool H in S))


    syntax Set ::= typeSet(K,K) [function] //(K, KSet)
    rule typeSet(TList(ListItem(TObject(ModName:K, A:K,B:K,C:K)) Rest:List), D:Set) => typeSet(TList(Rest), SetItem(A) D)
    rule typeSet(TList(.List), D:Set) => D

//    rule aloopCheck(Alpha:Map, D:List, G:Set, H:KItem, .K) => keys(Alpha) ~> H
//         requires notBool H in keys(Alpha)

    rule aloopCheck(.Map, .List, .Set, .K, .K,T:K, S:Set) => .K
//    rule aloopCheck(AList(Front:List ListItem(AObject(H,B:K)) C:List), D:List, G:Set, H:ConId) => aloopCheck(AList(C:List), ListItem(B) D, SetItem(B) G, B)


//    syntax KItem ::= TList(K) //list of T objects for every new type introduced by data and newtype
//    syntax KItem ::= TObject(K,K,K) //(type name, entire list of poly type vars, list of inner T pieces)
//    syntax KItem ::= InnerTPiece(K,K,K,K,K) //(type constructor, poly type vars, argument sorts, entire constr block, type name)

//Make sure argument sorts [U] [W,V] are in the set of keys of alpha and targets of T, (keys of T)

    syntax KItem ::= argSortCheck(K,K,K) [function] //(T,AlphaMap)

    rule <k> checkArgSortsAreTargets
             => argSortCheck(T,A,typeSet(T,.Set)) ...</k>
         <tempAlphaMap> A:K </tempAlphaMap>
         <tempT> T:K </tempT>
    
    rule argSortCheck(TList(ListItem(TObject(ModName:K, A:K,B:K,ListItem(InnerTPiece(C:K,D:K,ListItem(Arg:KItem) ArgsRest:List,E:K,F:K)) InnerRest:List)) TListRest:List),AlphaMap:Map,Tset:Set) => argSortCheck(TList(ListItem(TObject(ModName,A,B,ListItem(InnerTPiece(C,D,ArgsRest,E,F)) InnerRest)) TListRest),AlphaMap,Tset)
         requires ((Arg in keys(AlphaMap)) orBool (Arg in Tset))

    rule argSortCheck(TList(ListItem(TObject(ModName:K,A:K,B:K,ListItem(InnerTPiece(C:K,D:K,ListItem(Arg:KItem) ArgsRest:List,E:K,F:K)) InnerRest:List)) TListRest:List),AlphaMap:Map,Tset:Set) => error
         requires (notBool ((Arg in keys(AlphaMap)) orBool (Arg in Tset)))

    rule argSortCheck(TList(ListItem(TObject(ModName:K,A:K,B:K,ListItem(InnerTPiece(C:K,D:K,.List,E:K,F:K)) InnerRest:List)) TListRest:List),AlphaMap:Map,Tset:Set) => argSortCheck(TList(ListItem(TObject(ModName,A,B,InnerRest)) TListRest),AlphaMap,Tset)

    rule argSortCheck(TList(ListItem(TObject(ModName:K,A:K,B:K,.List)) TListRest:List),AlphaMap:Map,Tset:Set) => argSortCheck(TList(TListRest),AlphaMap,Tset)

    rule argSortCheck(TList(.List),AlphaMap:Map,Tset:Set) => .K

//NEED TO CHECK all the polymorphic parameters from right appear on left. RIGHT SIDE ONLY
//NEED TO CHECK UNIQUENESS FOR POLY PARAM ON LEFT SIDE ONLY

    syntax KItem ::= parCheck(K,K) [function] //(T,AlphaMap)
    syntax KItem ::= makeTyVarList(K,K,K) [function] //(TyVars, NewList)
    syntax KItem ::= lengthRet(K,K,K) [function]

    rule <k> checkParUsed
             => parCheck(T,.K) ...</k>
         <tempT> T:K </tempT>

    //rule makeParLists(TList(ListItem(TObject(A:K,ListItem(Arg:KItem) PolyList:List,C:K)) Rest:List),Tlist:List,Tset:Set) => makeParLists(TList(ListItem(TObject(A,PolyList,C)) Rest),ListItem(Arg) Tlist,SetItem(Arg) Tset)
    rule parCheck(TList(ListItem(TObject(ModName:K,A:K,B:K,C:K)) Rest:List),.K) => parCheck(TList(ListItem(TObject(ModName,A:K,B:K,C:K)) Rest:List),makeTyVarList(B,.List,.Set))

    rule parCheck(TList(ListItem(TObject(ModName:K,A:K,B:K,ListItem(InnerTPiece(C:K,ListItem(Par:KItem) ParRest:List,D:K,E:K,F:K)) InnerRest:List)) Rest:List),NewSet:Set) =>
         parCheck(TList(ListItem(TObject(ModName,A,B,ListItem(InnerTPiece(C,ParRest,D,E,F)) InnerRest)) Rest),NewSet)
            requires Par in NewSet

    rule parCheck(TList(ListItem(TObject(ModName:K,A:K,B:K,ListItem(InnerTPiece(C:K,ListItem(Par:KItem) ParRest:List,D:K,E:K,F:K)) InnerRest:List)) Rest:List),NewSet:Set) => error
            requires notBool (Par in NewSet)

    rule parCheck(TList(ListItem(TObject(ModName:K,A:K,B:K,ListItem(InnerTPiece(C:K,.List,D:K,E:K,F:K)) InnerRest:List)) Rest:List),NewSet:Set) =>
         parCheck(TList(ListItem(TObject(ModName,A,B,InnerRest)) Rest),NewSet)

    rule parCheck(TList(ListItem(TObject(ModName:K,A:K,B:K,.List)) Rest:List),NewSet:Set) =>
         parCheck(TList(Rest),NewSet)

    rule parCheck(TList(.List),NewSet:Set) => .K

    rule makeTyVarList('typeVars(A:K,,Rest:K),NewList:List,NewSet:Set) => makeTyVarList(Rest, ListItem(A) NewList, SetItem(A) NewSet)

    rule makeTyVarList(.TyVars,NewList:List,NewSet:Set) => lengthRet(size(NewList),size(NewSet),NewSet)

    rule lengthRet(A:Int, B:Int, C:K) => C
                    requires A ==Int B

    rule lengthRet(A:Int, B:Int, C:K) => error
                    requires A =/=Int B
    
    //rule argSortCheck(TList(ListItem(TObject(A:K,B:K,C:K)

////////////////////////////////////////////////////////////////////////////////////////////////////////////////////////

    // STEP 3 Transform T into beta

    syntax KItem ::= "startTTransform"
    syntax KItem ::= "constructDelta"
    syntax KItem ::= "constructBeta"

    rule <k> startTTransform
             => constructDelta
                ~> (constructBeta) ...</k>

    rule <k> constructDelta
             => makeDelta(T,.Map) ...</k>
         <tempT> T:K </tempT>

    syntax KItem ::= makeDelta(K,Map) [function] //(T,Delta)
    syntax KItem ::= newDelta(Map) //Delta
    syntax KItem ::= newBeta(Map) //beta
    syntax List ::= retPolyList(K,List) [function] //(T,Delta)

    rule makeDelta(TList(ListItem(TObject(ModName:K,A:K,Polys:K,C:K)) Rest:List),M:Map) =>
         makeDelta(TList(Rest),M[ModPlusType(ModName,A) <- size(retPolyList(Polys,.List))])
    rule makeDelta(TList(.List),M:Map) => newDelta(M)

    rule retPolyList('typeVars(A:K,,Rest:K),NewList:List) => retPolyList(Rest, ListItem(A) NewList)
    rule retPolyList(.TyVars,L:List) => L

    rule <k> newDelta(M:Map)
             => .K ...</k>
         <tempDelta> OldDelta:K => M </tempDelta>

    rule <k> constructBeta
             => makeBeta(T,.Map) ...</k>
         <tempT> T:K </tempT>

    syntax KItem ::= makeBeta(K,Map) [function] //(T,Beta,Delta)

    rule makeBeta(TList(ListItem(TObject(ModName:K,A:K,B:K,ListItem(InnerTPiece(Con:K,H:K,D:K,E:K,F:K)) InnerRest:List)) Rest:List),Beta:Map) =>
         makeBeta(TList(ListItem(TObject(ModName,A,B,InnerRest)) Rest),Beta[ModPlusType(ModName,Con) <- betaParser(E,B,A)])
    rule makeBeta(TList(ListItem(TObject(ModName:K,A:K,B:K,.List)) Rest:List),Beta:Map) =>
         makeBeta(TList(Rest),Beta)
    rule makeBeta(TList(.List),Beta:Map) =>
         newBeta(Beta)
//    rule makeBeta(TList(ListItem(TObject(ModName:K,A:K,B:K,ListItem(InnerTPiece(C:K,H:K,D:K,E:K,F:K)) InnerRest:List)) Rest:List),Beta:Map) =>
//         makeBeta(TList(ListItem(TObject(ModName,A,B,InnerRest)) Rest),Beta)

    syntax KItem ::= betaParser(K,K,K) [function] //(Tree Piece,NewSyntax,Parameters,Constr)
    syntax Set ::= getTyVarsRHS(K,List) [function]

    syntax KItem ::= forAll(Set,K)
    syntax KItem ::= funtype(K,K)

    syntax Set ::= listToSet(List, Set) [function]

    rule listToSet(ListItem(A:KItem) L:List, S:Set) => listToSet(L, SetItem(A) S)
    rule listToSet(.List, S:Set) => S


//if optbangAtypes, need to see if first variable is a typecon
//if its a typecon then need to go into Delta and see the amount of parameters it has
//then count the number of optbangAtypes after the typecon
    rule betaParser('constrCon(A:K,, B:K), Par:K, Con:K) => forAll(getTyVarsRHS(B,.List), betaParser(B, Par, Con))
    rule betaParser('optBangATypes('optBangAType('emptyBang(.KList),, 'atypeTyVar(Tyv:K)),, Rest:K), Par:K, Con:K) => funtype(Tyv, betaParser(Rest, Par, Con))
    rule betaParser('optBangATypes('optBangAType('emptyBang(.KList),, 'baTypeCon(A:K,, B:K)),, Rest:K), Par:K, Con:K) => funtype('baTypeCon(A:K,, B:K), betaParser(Rest, Par, Con))
    rule betaParser('optBangATypes('optBangAType('emptyBang(.KList),, 'atypeGTyCon(Tyc:K)),, Rest:K), Par:K, Con:K) => funtype(Tyc, betaParser(Rest, Par, Con))
    rule betaParser(.OptBangATypes, Par:K, Con:K) => 'simpleTypeCon(Con,, Par)
//    rule betaParser('optBangATypes('optBangAType('emptyBang(.KList),, 'atypeGTyCon(Tyc:K)),, Rest:KItem)) => getTypeVars(A) getTypeVars(Rest)
//    rule getTypeVars('optBangAType('emptyBang(.KList),, Rest:K)) => getTypeVars(Rest)
//    rule getTypeVars('atypeGTyCon(A:K)) => .List
//    rule getTypeVars('atypeTyVar(A:K)) => ListItem(A)
//    rule getTypeVars(.OptBangATypes) => .List

    rule getTyVarsRHS(.OptBangATypes,Tylist:List) => listToSet(Tylist, .Set)

    rule <k> newBeta(M:Map)
             => .K ...</k>
         <tempBeta> OldBeta:K => M </tempBeta>

////////////////////////////////////////////////////////////////////////////////////////////////////////////////////////

//    syntax KItem ::= "insertAlphaBetaStar"

    syntax KItem ::= insertABRec(K,List)
    syntax KItem ::= insertAB(K)

    rule <k> insertAlphaBetaStar => insertABRec(Mod, Imp) ...</k>
         <tempModule> Mod:KItem </tempModule>
         <importTree> Imp:List </importTree>

    rule <k> insertABRec(Node:KItem, ListItem(impObject(B:KItem,Node)) I:List) => insertAB(B) ~> insertABRec(Node, I) ...</k>

    rule <k> insertABRec(Node:KItem, ListItem(impObject(B:KItem,C:KItem)) I:List) => insertABRec(Node, I) ...</k>
             requires Node =/=K C

    rule <k> insertAB(B) => .K ...</k>

         <tempAlphaStar> Alph:KItem </tempAlphaStar>
         <tempBetaStar> Bet:KItem </tempBetaStar>

         <moduleName> 'moduleName(B) </moduleName>
         <moduleImpAlphas> ImpAlphas:List => ListItem(Alph) ImpAlphas </moduleImpAlphas>
         <moduleImpBetas> ImpBetas:List => ListItem(Bet) ImpBetas </moduleImpBetas>


endmodule
\end{lstlisting}