\chapter{Configuration}
K is used for defining a state machine and the K rules define the transition rules for the state machine. The configuration of the state machine is made up of K cells. The K cells contain the syntax data structure representing the code of the example program. They also contain the memory of the state machine. An actual state of the state machine in K is when the cells each have some term inside of them.

The following is the configuration of my Haskell semantics.
\begin{lstlisting}
requires "haskell-syntax.k"

module HASKELL-CONFIGURATION
    imports HASKELL-SYNTAX

    syntax KItem ::= "startImportRecursion"
    syntax KItem ::= callInit(K)
    //syntax KItem ::= initPreModule(K) [function]
    //syntax KItem ::= tChecker(K) [function]

    configuration 
        <T>
            <k> $PGM:ModuleList ~> startImportRecursion </k>
            <tempModule> .K </tempModule>
            <tempCode> .K </tempCode>
\end{lstlisting}

The $<k>$ cell is the cell that computation takes place in.
The abstract syntax tree is initially placed into the $<k>$ cell.
The command
\begin{lstlisting}
$PGM:ModuleList
\end{lstlisting}
means that the parsed tree appears in this cell and the sort that contains all other sorts is ModuleList.

$.K$ means that the cell is initially empty.

tempModule is the name of the current module. tempCode is the current code.

typeIterator is used for creating a fresh type variable for the inference algorithm. It has the current count of how many fresh type variables that were created.

\begin{lstlisting}
            <typeIterator> 1 </typeIterator>
\end{lstlisting}

\section{Alpha}
Alpha is a map of type renamings. So if a user declares

\begin{lstlisting}
data MyBool = TTrue
;type MyBooltwo = MyBool
\end{lstlisting}

Then MyBooltwo is a renaming of MyBool. In tempAlpha, an AObject is made. An AObject is a KItem with two children. One can be thought of as a Key and the other is the Value for a map. So MyBool $->$ MyBooltwo. However, we want to check and reject programs that have multiple renamings, so we cannot use a K Map which has idempotence. However, once we make this check, we can then use a K Map. This is what tempAlphaMap is.

.Map means that the cell starts with an empty map.

\begin{lstlisting}
            <tempAlpha> .K </tempAlpha>
            <tempAlphaMap> .Map </tempAlphaMap>
\end{lstlisting}

\section{Beta}

tempT contains all user defined datatypes. tempT is organized in such a way that makes context sensitive checks easy to perform. tempBeta contains all user defined datatypes organized so that type inference is easy to perform.
\begin{lstlisting}
            <tempBeta> .Map </tempBeta>
            <tempT> .K </tempT>
\end{lstlisting}

\section{Delta}

tempDelta contains the arity of the user defined dataTypes. So if a user defined datatype takes in two parameters, tempDelta will contain the number 2.

\begin{lstlisting}
            <tempDelta> .Map </tempDelta>
\end{lstlisting}

\section{Import Data Structure}

importTree, recurImportTree, and impTreeVMap contain the data necessary for the directed acyclic graph representing imports.

\begin{lstlisting}
            <importTree> .List </importTree>
            <recurImportTree> .List </recurImportTree>
            <impTreeVMap> .Map </impTreeVMap>
\end{lstlisting}

\section{Modules}

The modules cell contains all modules that were checked and inferred already. Multiplicity means that there can be multiple module cells.

\begin{lstlisting}
            <modules> //static information about a module
                <module multiplicity="*">
                    <moduleName> .K </moduleName>
                    <moduleAlphaStar> .K </moduleAlphaStar>
                    <moduleBetaStar> .K </moduleBetaStar>
                    <moduleImpAlphas> .List </moduleImpAlphas>
                    <moduleImpBetas> .List </moduleImpBetas>
                    <moduleCompCode> .K </moduleCompCode>
                    <moduleTempCode> .K </moduleTempCode>
                    <imports> .Set </imports>
                    <classes> //static information about a module
                        <class multiplicity="*">
                            <className> .K </className>
                        </class>
                    </classes>
                </module>
            </modules>
        </T>

endmodule
\end{lstlisting}